\section{Self-Suspending Sporadic Real-Time Task Models}
  
Self-suspending tasks can be classified into two models: \emph{dynamic} self-suspension and \emph{segmented} (or \emph{multi-segment}) self-suspension models. 
The dynamic self-suspension sporadic task model characterizes each task as a $4$-tuple $(C_i,S_i,T_i,D_i)$: $T_i$ denotes the minimum inter-arrival time of $\tau_i$, each job of $\tau_i$ has a relative deadline $D_i$,
$C_i$ denotes the upper bound on total execution time of each job of $\tau_i$,
and $S_i$ denotes the upper bound on total suspension time of each job of $\tau_i$. The multi-segment sporadic task suspending model characterizes the execution of a job of a task $\tau_i$ by  specifying its computation segments and suspension intervals are specified as an array
$(C_{i}^0,S_{i}^0,C_{i}^1,S_{i}^1,...,S_{i}^{m_i-2},C_{i}^{m_i-1})$ composed of $m_i$ computation segments separated by $m_i-1$ suspension intervals. 

From the system designer's perspective, the dynamic self-suspension model provides an easy way to specify self-suspending systems without considering the juncture of I/O access or computation offloading. 
However, from the analysis perspective, such a  dynamic model leads to quite pessimistic results in terms of schedulability since the location of suspensions within a job is oblivious. Therefore, if the suspending patterns are well-defined and characterized with known suspending intervals, the multi-segment self-suspension task model is more appropriate.   

\subsection{Examples of Dynamic Self-Suspension Model} 
  \textit{different program paths}
  \textit{self-suspension due to synchronizations}
  etc.
  
\subsection{Examples of Segmented Self-Suspension Model} 
  \textit{static execution patterns}
  \textit{multiprocessor synchronization with critical sections}
  etc.
  
  