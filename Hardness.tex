\section{Hardness Review of Self-Suspending Task Models}
This section reviews the hardness for designing scheduling algorithms and schedulability analysis of self-suspending task systems. 

\paragraph{Hardness for scheduling segmented self-suspending tasks}
Verifying the existence of a feasible schedule for segmented self-suspending task systems is proved to be ${\cal NP}$-hard in the strong sense in \cite{Ridouard_2004}. It is also shown that EDF and RM do not have any speedup factor bound in in \cite{Ridouard_2004} and \cite{RTSS-ChenL14}, respectively. 

The only results with speedup factor analysis for fixed-priority scheduling and dynamic priority scheduling can be found in \cite{RTSS-ChenL14} and \cite{WC16-suspend-DATE}. The analysis with speedup factor $3$ in \cite{RTSS-ChenL14} can be used for systems with at most one self-suspension interval per task in dynamic priority scheduling. The analysis with a bounded speedup factor in \cite{WC16-suspend-DATE} can be used for fixed-priority and dynamic-priority systems with any number of self-suspension intervals per task. However, the speedup factor in \cite{WC16-suspend-DATE} depends on, and grows quadratically with respect to the number of self-suspension intervals. Therefore, it can only be used for a few number of suspension intervals. 

With respect to this problem, there was no lower bound (with respect to the speedup factors) of this scheduling problem. 

\paragraph{Hardness for scheduling dynamic self-suspending tasks}
The complexity class for verifying the existence of a feasible schedule for dynamic self-suspending task systems was unknown in the literature. The proof in \cite{Ridouard_2004} cannot be applied to this case. 


\begin{theorem}
The speed-up factor for RM, DM, and LM scheduling is $\infty$.
\end{theorem}
\begin{proof}
Consider the following implicit-deadline task set with one SSS task and one sporadic task:
\begin{itemize}
 \setlength\itemsep{0em}
\item $C_1=1-2\epsilon$, $S_1=0$, $T_1=1$
\item  $C_2=\epsilon$, $S_2=T-1-\epsilon$, $T_2=T$
\end{itemize}

where $T$ is any natural number larger than $1$ and $\epsilon$ can be arbitrary small.
It is clear that this task set is schedulable if we assign higher priority to
task $\tau_2$ than task $\tau_1$.

Under either RM, DM, and LM scheduling, task $\tau_1$ has higher priority than task $\tau_2$. 
Thanks to the harmonic system, the schedulability of task $\tau_2$ can be analyzed by examining the demand from task $\tau_1$ together with its upper bound on total execution time and suspension length at its deadline, as the two tasks release simultaneously.
Thus, jobs of $\tau_2$ will miss deadlines since $T( 1-2\epsilon)+\epsilon+(T-1-\epsilon) > T$.

In order to be schedulable upon this system under RM, DM, or LM on $\alpha$-speed processor the following must hold\footnote{On a speed-$\alpha$ uniprocessor the worst-case execution times $C_{i}$ become $\frac{C_{i}}{\alpha}$. However, $S_i$ remains the same.}:
\begin{equation*}
\frac{T( 1-2\epsilon)+\epsilon}{\alpha}+(T-1-\epsilon) \le T
\end{equation*}
After reformulation, we have $\alpha\ge \frac{T( 1-2\epsilon)+\epsilon}{1+\epsilon}$. Thus, $\alpha\rightarrow \infty$ as $T \rightarrow \infty$.
\end{proof}


   
\begin{itemize}
\item \textbf{The complexity class of scheduling policies}
\item \textbf{Open problems for schedulability analysis, etc.}
\end{itemize}
  
\section{Rule of Thumb to Handle Self-Suspending Task Systems}