\section{Hardness Review of Self-Suspending Task Models}
This section reviews the hardness for designing scheduling algorithms and schedulability analysis of self-suspending task systems. Table \ref{table:complexity} summarizes the complexity classes of the corresponding problems, in which the complexity problems are reviewed according the considered task models (\ie, segmented or dynamic self-suspending models) and the scheduling strategies (\ie, fixed- or dynamic-priority scheduling). Notably, for self-suspending task systems, only the complexity class for verifying the existence of a feasible schedule for segmented tasks is proved in the literature~\cite{Ridouard_2004}, most corresponding problems are still open.

\begin{table}
\centering
    \begin{tabular}{|c|c|c|c|c|}
 \hline
        Task Model & Scheduling & \multicolumn{3}{c|}{Schedulability Test} \\
        \hline
        &  & Fixed-Priority & \multicolumn{2}{c|}{Dynamic-Priority}\\
        &  & Scheduling     & \multicolumn{2}{c|}{Scheduling}\\
        \cline{3-5}    
        Segmented Self- & ${\cal NP}$-hard in &  & Constrained & Implicit\\
        Suspension  & the strong & at least weakly   & Deadlines   & Deadlines \\
        \cline{4-5}
        Models & sense \cite{Ridouard_2004} & ${\cal NP}$-hard & at least co${\cal NP}$- & co${\cal NP}$-hard \\
        &  & & hard in the & in the strong\\
        & & & strong sense & sense\\
        \hline
        Dynamic Self- & \multirow{3}{*}{Unknown} & & at least co${\cal NP}$- & \multirow{3}{*}{Unknown}\\
        Suspension & & at least weakly &hard in the & \\
        models & & ${\cal NP}$-hard & strong sense & \\
        \hline
    \end{tabular}
    \caption{The complexity classes of scheduling and schedulability analysis for self-suspending tasks}
    \label{table:complexity}
\end{table}

\subsection{Hardness for Scheduling Segmented Self-Suspending Tasks}
Verifying the existence of a feasible schedule for segmented self-suspending task systems is proved to be ${\cal NP}$-hard in the strong sense in \cite{Ridouard_2004}. It is also shown that EDF and RM do not have any speedup factor bound in in \cite{Ridouard_2004} and \cite{RTSS-ChenL14}, respectively. 

The only results with speedup factor analysis for fixed-priority scheduling and dynamic priority scheduling can be found in \cite{RTSS-ChenL14} and \cite{WC16-suspend-DATE}. The analysis with speedup factor $3$ in \cite{RTSS-ChenL14} can be used for systems with at most one self-suspension interval per task in dynamic priority scheduling. The analysis with a bounded speedup factor in \cite{WC16-suspend-DATE} can be used for fixed-priority and dynamic-priority systems with any number of self-suspension intervals per task. However, the speedup factor in \cite{WC16-suspend-DATE} depends on, and grows quadratically with respect to the number of self-suspension intervals. Therefore, it can only be \emph{practically} used when there are only a few number of suspension intervals per task. The scheduling policy used in \cite{WC16-suspend-DATE} is \emph{laxity-monotonic} (LM) scheduling, which assigns the highest priority to the task with the least laxity, that is, $D_i-S_i$.

With respect to this scheduling problem, there was no theoretical lower bound (with respect to the speedup factors) of this scheduling problem. 


The above analysis also implies that the priority assignment in fixed-priority scheduling should be carefully designed. Traditional approaches like RM or EDF do not work very well. LM may work for a few self-suspending intervals, but how to perform the optimal priority assignment is an open problem.


\subsection{Hardness for Scheduling Dynamic Self-Suspending Tasks}
The complexity class for verifying the existence of a feasible schedule for dynamic self-suspending task systems is unknown in the literature. The proof in \cite{Ridouard_2004} cannot be applied to this case. 
It is proved in \cite{huangpass:dac2015} that the speed-up factor for RM, DM, and LM scheduling is $\infty$. Here, we repeat the example in \cite{huangpass:dac2015}. Consider the following implicit-deadline task set with one self-suspending task and one sporadic task:
\begin{itemize}
 \setlength\itemsep{0em}
\item $C_1=1-2\epsilon$, $S_1=0$, $T_1=1$
\item  $C_2=\epsilon$, $S_2=T-1-\epsilon$, $T_2=T$
\end{itemize}
where $T$ is any natural number larger than $1$ and $\epsilon$ can be arbitrary small.

It is clear that this task set is schedulable if we assign higher priority to
task $\tau_2$.
Under either RM, DM, and LM scheduling, task $\tau_1$ has higher priority than task $\tau_2$. It was proved in \cite{huangpass:dac2015} that this example has a speed-up factor $\infty$ when $\epsilon$ is close to $0$.

There is no upper bound of this problem in the most general case. The analysis in \cite{huangpass:dac2015} for a speedup factor $2$ uses a trick to compare the speedup factor with respect to the \emph{optimal fixed-priority schedule} instead of the \emph{optimal schedule}. There is no proof or evident to show that this factor $2$ is also the factor when the reference is the \emph{optimal schedule}.

With respect to this problem, there was no theoretical lower bound (with respect to the speedup factors) of this scheduling problem. 


The above analysis also implies that the priority assignment in fixed-priority scheduling should be carefully designed. Traditional approaches like RM or EDF do not work very well. LM also does not work well. The priority assignment used in \cite{huangpass:dac2015} is based on the optimal-priority algorithm (OPA) from Audsley \cite{audsley-1993} with an OPA-compatible schedulability analysis. However, since the schedulability test used in \cite{huangpass:dac2015}  is not exact, the priority assignment is also not the optimal solution.
Finding the optimal priority assignment here is also an open problem.


\subsection{Hardness for Schedulability Tests for Segmented Self-Suspension}
\paragraph{Preemptive Fixed-Priority Scheduling:}   
For this case, the complexity class of verifying whether the worst-case response time is no more than the relative deadline is \emph{unknown} up to now. The evidence provided in \cite{ecrts15nelissen} also suggests that this problem may be very difficult even for a task system with \emph{only one self-suspending task}. The solution in \cite{ecrts15nelissen}  requires exponential time complexity for $n-1$ sporadic tasks and 1 self-suspending task. The other solutions \cite{Huang:multiseg}\cite{PH:rtss98} require pseudo-polynomial time complexity but are only sufficient schedulability tests.

The lack of something like the critical instant theorem in the ordinary sporadic task systems to reduce the search space of the worst-case behaviour has led to the complexity explosion to test exponential combinations of release patterns. The complexity class is at least as hard as the ordinary sporadic task systems under fixed-priority scheduling, which is weakly ${\cal NP}$-hard  \cite{EisenbrandR08}. Whether the problem (with segmented self-suspension) is ${\cal NP}$-hard in the strong sense is an open problem.

\paragraph{Preemptive Dynamic-Priority Scheduling:} 
For this case, if the task systems are with constrained deadlines, i.e., $D_i \leq T_i$, the complexity class of this problem is at least co${\cal NP}$-hard in the strong sense, since a special case of this problem is co${\cal NP}$-complete in the strong sense \cite{DBLP:conf/ecrts/Ekberg015}. It has been proved in \cite{DBLP:conf/ecrts/Ekberg015} that verifying uniprocessor feasibility of sporadic tasks with constrained deadlines is strongly co${\cal NP}$-complete.  Therefore, when we consider constrained-deadline self-suspending task systems, the complexity class is at least co${\cal NP}$-hard in the strong sense.

It is also not difficult to see that the implicit-deadline case is also at least co${\cal NP}$-hard.  A special case of segmented self-suspending task system is to allow a task $\tau_i$ having exactly one self-suspension interval with a \emph{fixed} length $S_i$ and one computation segment with WCET $C_i$. Therefore, the relative deadline of the computation segment of task $\tau_i$ (after it is released to be scheduled) is $D_i = T_i-S_i$. Therefore, the implicit-deadline segmented self-suspending task system is equivalent to a constrained-deadline task system, which is co${\cal NP}$-complete in the strong sense. Since a special case of the problem is co${\cal NP}$-complete in the strong sense, the problem is co${\cal NP}$-hard in the strong sense.


\subsection{Hardness for Schedulability Tests for Dynamic Self-Suspension}
\paragraph{Preemptive Fixed-Priority Scheduling:}   

Similarly, for this case, with dynamic self-suspension, the complexity class of verifying whether the worst-case response time is no more than the relative deadline is \emph{unknown} up to now. There is \emph{no exact} schedulability analysis for this problem up to now. The solutions in \cite{Liu:2000:RS:518501}\cite{LiuChen:rtss2014}\cite{huangpass:dac2015} are only sufficient schedulability tests. 

The lack of something like the critical instant theorem and the dynamics of the dynamic self-suspending behaviour have constrained the current researches to provide exact schedulability tests. The complexity class is at least as hard as the ordinary sporadic task systems under fixed-priority scheduling, which is weakly ${\cal NP}$-hard \cite{EisenbrandR08}. Whether the problem (with dynamic self-suspension) is ${\cal NP}$-hard in the strong sense is an open problem.

\paragraph{Preemptive Dynamic-Priority Scheduling:} 
For this case, if the task systems are with constrained deadlines, i.e., $D_i \leq T_i$, similarly, the complexity class of this problem is at least co${\cal NP}$-hard in the strong sense, since a special case of this problem is co${\cal NP}$-complete in the strong sense \cite{DBLP:conf/ecrts/Ekberg015}. For implicit-deadline self-suspending task systems, the schedulability test problem is not well-defined, since there is no clear scheduling policy that can be applied and tested. Therefore, we would conclude this as an open problem.

  
\section{Rule of Thumb to Handle Self-Suspending Task Systems}