\mychapter{Computational Complexity and Approximations}
\label{sec:hardness}
This \mysectionrefnormal{} reviews the difficulty of designing scheduling algorithms and schedulability analyses of self-suspending task systems. Table \ref{table:complexity} summarizes the computational complexity classes of the corresponding problems, in which the complexity problems are reviewed according to the considered task models (\ie, segmented or dynamic self-suspending models) and the scheduling strategies (\ie, fixed- or dynamic-priority scheduling). 

%Notably, for self-suspending task systems, only the complexity class for verifying the existence of a feasible schedule for segmented tasks is proved in the literature~\cite{Ric03,Ridouard_2004}, most corresponding problems are still open.

\begin{table}[t]
\centering
    \begin{tabular}{|c|c|c|c|c|}
 \hline
        Task Model & Feasibility & \multicolumn{3}{c|}{Schedulability} \\
        \hline
        &  & Fixed-Priority & \multicolumn{2}{c|}{Dynamic-Priority}\\
        &  & Scheduling     & \multicolumn{2}{c|}{Scheduling}\\
        \cline{3-5}    
        Segmented & ${\cal NP}$-hard in &  & Constrained & Implicit\\
         Self-Suspension  & the strong &   & Deadlines   & Deadlines \\
        \cline{4-5}
        Models & sense \cite{Ric03,Ridouard_2004} & co${\cal NP}$-hard& co${\cal NP}$-hard & co${\cal NP}$-hard \\
        &  & in the strong & in the strong & in the strong\\
        & & sense \cite{DBLP:conf/rtns/MohaqeqiE016,RTSS2016-suspension}& sense \cite{RTSS2016-suspension}& sense \cite{RTSS2016-suspension}\\
        \hline
        Dynamic & \multirow{3}{*}{unknown} & (at least) ${\cal NP}$-hard & co${\cal NP}$-hard & \multirow{3}{*}{unknown}\\
         Self-Suspension & &  in the weak&in the strong& \\
        Models & & sense \cite{DBLP:conf/rtss/Ekberg017}& sense \cite{RTSS2016-suspension}& \\
        \hline
    \end{tabular}
    \vskip 0.2in
    \caption{The computational complexity classes of scheduling and schedulability analysis for self-suspending tasks}
    \label{table:complexity}
\end{table}

\mysection{Computational Complexity of Designing Scheduling Policies}
\label{sec:complexity-scheduling}

We first present the computational complexity of designing scheduling policies for both self-suspending task models considered in this report.

\mysubsection{Segmented Self-Suspending Tasks}
Verifying the existence of a feasible schedule for segmented self-suspending task systems is proved to be ${\cal NP}$-hard in the strong sense in \cite{Ridouard_2004} for implicit-deadline tasks with at most one self-suspension per task. For this model, it is also shown that EDF and RM do not have any speedup factor bound in \cite{Ridouard_2004} and \cite{RTSS-ChenL14}, respectively. For the generalization of the segmented self-suspension model to multi-threaded tasks (\ie, every task is defined by a Directed Acyclic Graph (DAG) with edges labelled by suspension delays), the feasibility problem is also known to be  ${\cal NP}$-hard in the strong sense  \cite{Ric03} even if all sub-jobs have unit execution times. Up to now, there is no known theoretical lower bound with respect to the speedup factors for this scheduling problem.

 The only results with speedup factor analysis for fixed-priority scheduling and dynamic-priority scheduling can be found in \cite{RTSS-ChenL14,WC16-suspend-DATE,Bruggen16RTNS}. The analysis with a speedup factor of $3$ in \cite{RTSS-ChenL14,Bruggen16RTNS} can be used for systems with at most one self-suspension interval per task under dynamic-priority scheduling. The analysis with a bounded speedup factor in \cite{WC16-suspend-DATE} can be used for fixed-priority and dynamic-priority systems with any number of self-suspension intervals per task. 
 The scheduling policy used in \cite{WC16-suspend-DATE} is \emph{suspension laxity-monotonic} (SLM) scheduling, which assigns the highest priority to the task with the least suspension laxity, defined as $D_i-S_i$.
However, the speedup factor of SLM  depends on the number of self-suspension intervals, and grows quadratically with respect to it.
%Therefore, it can be \emph{practically} used only  when there are  a few number of suspension intervals per task.



The above analysis also implies that the priority assignment in dynamic-priority and fixed-priority scheduling should be carefully designed. Traditional approaches like RM or EDF do not work very well. SLM may work well for a few self-suspension intervals, but how to perform the optimal priority assignment is an open problem. Such difficulty comes from scheduling anomalies that may occur at run-time. An example is provided in \cite{Ridouard_2004}  to show that reducing execution times or self-suspension delays can result in deadline misses under EDF (\ie, EDF is no longer sustainable). This latter result can be easily extended to fixed-priority scheduling policies (\ie, RM and DM). Lastly, in \cite{RidouardR06}, it is proved that no deterministic online scheduler can be optimal if the real-time tasks are allowed to suspend themselves.



\mysubsection{Dynamic Self-Suspending Tasks}
The computational complexity of verifying the existence of a feasible schedule for dynamic self-suspending task systems is unknown. The proof in \cite{Ridouard_2004} cannot be applied to this case. 
It is proved in \cite{huangpass:dac2015} that the speedup factor for RM, DM, and suspension laxity monotonic (SLM) scheduling is $\infty$. Here, we repeat the example in \cite{huangpass:dac2015}. Consider the following implicit-deadline task set with one self-suspending task and one sporadic task:
\begin{itemize}
 \setlength\itemsep{0em}
\item $C_1=1-2\epsilon$, $S_1=0$, $T_1=1$
\item  $C_2=\epsilon$, $S_2=T-1-\epsilon$, $T_2=T$
\end{itemize}
where $T$ is any natural number larger than $1$ and $\epsilon$ can be arbitrary small. It is clear that this task set is schedulable if we assign the highest priority to
task $\tau_2$. Under either RM, DM, and SLM scheduling, task $\tau_1$ has higher priority than task $\tau_2$. It was proved in \cite{huangpass:dac2015} that this example has a speedup factor $\infty$ when $\epsilon$ approaches $0$.


There is no upper bound of this problem in the most general case. The analysis in \cite{huangpass:dac2015} for a speedup factor $2$ uses a trick to compare the speedup factor with respect to the \emph{optimal fixed-priority schedule} instead of the \emph{optimal schedule}.  The priority assignment used in \cite{huangpass:dac2015} is based on the optimal-priority algorithm (OPA) from Audsley \cite{audsley-1993} with an OPA-compatible schedulability analysis. However, since the schedulability test used in \cite{huangpass:dac2015} is not exact, the priority assignment is also not the optimal solution.  Finding the optimal priority assignment for fixed-priority scheduling is still an open problem.

For dynamic self-suspending task systems, as shown in \cite{RTSS2016-suspension}, the speedup factor for any FP preemptive scheduling, compared to the optimal   schedules, is not bounded by a constant if the suspension time cannot be reduced by speeding up. Such a statement of unbounded speedup factors was proved in \cite{RTSS2016-suspension} for earliest-deadline-first (EDF), least-laxity-first (LLF), and earliest-deadline-zero-laxity (EDZL) scheduling algorithms. How to design good schedulers with a constant speedup factor remains as an open problem.




\mysection{Computational Complexity of Schedulability Tests}

We now present the computational complexity of schedulability tests for both self-suspending task models considered in this report.


\mysubsection{Segmented Self-Suspending Tasks}
\paragraph{Preemptive Fixed-Priority Scheduling:}   
In this case, the computational complexity of schedulability tests is $co{\cal NP}$-hard in the strong sense even when the lowest priority task has at least two self-suspension intervals and the higher-priority sporadic tasks do not suspend themselves \cite{RTSS2016-suspension,DBLP:conf/rtns/MohaqeqiE016}. The computational complexity analysis holds for both implicit-deadline and constrained-deadline task systems, when the priority assignment is given.  Moreover, validating whether there exists a feasible priority assignment is $co{\cal NP}$-hard in the strong sense for constrained-deadline segmented self-suspending task systems.

\paragraph{Preemptive Dynamic-Priority Scheduling:} 
In this case, if the task systems have constrained deadlines, \ie, $D_i \leq T_i$, the computational complexity of this problem is at least co${\cal NP}$-hard in the strong sense, since a special case of this problem is co${\cal NP}$-complete in the strong sense \cite{DBLP:conf/ecrts/Ekberg015}. It has been proved in \cite{DBLP:conf/ecrts/Ekberg015} that verifying uniprocessor feasibility of ordinary sporadic tasks with constrained deadlines is strongly co${\cal NP}$-complete.  Therefore, when we consider constrained-deadline self-suspending task systems, the complexity class is at least co${\cal NP}$-hard in the strong sense.

It is also not difficult to see that the implicit-deadline case is also at least co${\cal NP}$-hard.  A special case of the segmented self-suspending task system is to allow each task $\tau_i$ to have exactly one self-suspension interval with a \emph{fixed} length $S_i$ and one computation segment with WCET $C_i$.  Therefore, the relative deadline of the computation segment of task $\tau_i$ (after it is released to be scheduled) is $D_i = T_i-S_i$. 
For such a special case, self-suspension of a task is equivalent to a release offset of $S_i$. Therefore, there is no need to consider any self-suspension behavior any further. Scheduling in such a scenario is equivalent to ordinary constrained-deadline sporadic real-time task systems, in which preemptive EDF is optimal.
It has been proved in \cite{DBLP:conf/ecrts/Ekberg015} that verifying uniprocessor feasibility of ordinary sporadic tasks with constrained deadlines is strongly co${\cal NP}$-complete.
By the above discussions, any ordinary constrained-deadline sporadic task system can be converted to a corresponding implicit-deadline segmented self-suspending task system, and their exact schedulability tests for EDF scheduling are identical. Since a special case of the problem is co${\cal NP}$-complete in the strong sense, the problem is co${\cal NP}$-hard in the strong sense.


\mysubsection{Dynamic Self-Suspending Tasks}
\paragraph{Preemptive Fixed-Priority Scheduling:}   
In this case, the complexity class is at least as hard as that in the ordinary sporadic task systems under fixed-priority scheduling. It is shown in \cite{EisenbrandR08} that the response time analysis is at least weakly ${\cal NP}$-hard and the complexity class of the schedulability test problem is shown weakly ${\cal NP}$-complete by Ekberg and Wang~\cite{DBLP:conf/rtss/Ekberg017}. Therefore, the schedulability test problem for self-suspending task systems under fixed-priority scheduling is at least weakly ${\cal NP}$-hard.


The computational complexity due to the additional dynamic self-suspending behavior is in general \emph{unknown} up to now.  The only exception is the special case mentioned in Section~\ref{sec:existing-exact-special} when there is only one dynamic self-suspending sporadic task assigned to the lowest priority and the higher-priority tasks are ordinary sporadic tasks. That is, the computational complexity of this special case remains the same as that of non-self-suspending sporadic task systems. Whether the problem (with dynamic self-suspension) is ${\cal NP}$-hard in the weak or strong sense is an open problem.

\paragraph{Preemptive Dynamic-Priority Scheduling:} 
If the task systems have constrained deadlines, \ie, $D_i \leq T_i$, the computational complexity class of this problem is at least co${\cal NP}$-hard in the strong sense, since the computational complexity for testing the schedulability of an ordinary sporadic task system under the optimal dynamic-priority scheduling strategy, \ie, EDF, is co${\cal NP}$-complete in the strong sense \cite{DBLP:conf/ecrts/Ekberg015}. For implicit-deadline self-suspending task systems, the schedulability test problem is not well-defined, since there is no clear scheduling policy that can be applied and tested.  Even for the well-known dynamic-priority scheduling strategies like EDF, LLF, EDZL, and their variances as mentioned at the end of Section~\ref{sec:complexity-scheduling},  the computational complexity of schedulability tests and how to perform exact schedulability tests are both unknown for implicit-deadline self-suspending task systems.
%Therefore, we would conclude this as an open problem.



%%% Local Variables:
%%% mode: latex
%%% TeX-master: "JRTS/JRTS.tex"
%%% End:


  
