\iftechreport
\chapter*{Abstract}
\else
\begin{abstract}
\fi


  In general computing systems, a job (process/task) may suspend itself
  whilst it is waiting for some activity to complete, \eg, an accelerator
  to return required data or results from the offloaded computation.
  For real-time embedded systems, such self-suspension can cause
  substantial performance/schedulability degradation. This has led to the
  investigation of the impact of self-suspension behavior on
  timing predictability, with many results reported since 1990.

  This paper reviews the design and analysis of scheduling algorithms
  and schedulability tests for self-suspending tasks in real-time systems. 
  We report that a number of these existing approaches are flawed.  
  As a result, we provide (1) a systematic description
  of how self-suspending tasks can be handled in both soft and
  hard real-time systems; (2) an explanation of the existing misconceptions 
  and their potential remedies; (3) an assessment of the influence of 
  such flawed analysis on partitioned multiprocessor fixed-priority scheduling when tasks
  synchronize access to shared resources; and (4) a computational
  complexity analysis for different self-suspension task models.

  In summary, this paper provides a state-of-art review of existing real-time
  analysis of self-suspending tasks to provide a correct platform on which future
  research can be built. 

\iftechreport
\vfill
\else
\end{abstract}
\fi




%%% Local Variables:
%%% mode: latex
%%% TeX-master: "JRTS/JRTS.tex"
%%% End: