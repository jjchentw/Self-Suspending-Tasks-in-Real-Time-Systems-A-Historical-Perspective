\iftechreport
\chapter*{Abstract}
\else
\begin{abstract}
\fi


  In general computing systems, a job (process/task) may suspend itself
  whilst it is waiting for some activity to complete,~\eg, an accelerator
  to return data. % or results from offloaded computation.
  In real-time systems, such self-suspension can cause
  substantial performance/schedulability degradation. This observation, first made in 1988, has led to the
  investigation of the impact of self-suspension on
  timing predictability, and many relevant results have been published since.
  Unfortunately, as it has recently come to light, a number of the existing results are flawed.

  To provide a correct platform on which future research can be built,
  this paper reviews the state of the art in the design and analysis of scheduling algorithms
  and schedulability tests for self-suspending tasks in real-time systems. 
  We provide (1)~a systematic description
  of how self-suspending tasks can be handled in both soft and
  hard real-time systems; (2)~an explanation of the existing misconceptions 
  and their potential remedies; (3)~an assessment of the influence of 
  such flawed analyses on partitioned multiprocessor fixed-priority scheduling when tasks
  synchronize access to shared resources; and (4)~a discussion of the computational
  complexity of analyses for different self-suspension task models.

\iftechreport
\vfill
\else
\end{abstract}
\fi




%%% Local Variables:
%%% mode: latex
%%% TeX-master: "JRTS/JRTS.tex"
%%% End: