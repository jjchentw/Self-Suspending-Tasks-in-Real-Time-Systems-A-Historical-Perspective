
\mychapter{Existing Misconceptions in the State-of-the-Art}
\label{sec:misconceptions}

This \mysectionrefnormal{} explains several misconceptions in some existing results by presenting concrete examples to demonstrate their overstatements. These examples are constructed case by case. Therefore, each misconception will be explained by using one specific example. 

\mysection{Incorrect Quantifications of Jitter (Dynamic Self-Suspension)}
\label{sec:wrong-jitter-dynamic}


We first explain the misconceptions in the literature that quantify the jitter too optimistically for dynamic self-suspending
task systems under fixed-priority scheduling. 
To calculate the worst-case response time of the task $\tau_k$ under analysis, there have been several results in the literature, \ie, \cite{ECRTS-AudsleyB04,RTAS-AudsleyB04,RTCSA-KimCPKH95,MingLiRTCSA1994},  which propose to calculate the worst-case response time $R_k$ of task $\tau_k$ by finding the minimum $R_k$ with
\begin{equation}
R_k = C_k+ S_k+\sum_{\tau_i \in hp(k)}\ceiling{\frac{R_k+S_k}{T_i}} C_i,
\label{eq:dynamic-flawed}
\end{equation}
where the term $hp(k)$ is the set of the tasks with higher-priority levels than task $\tau_k$. 
This analysis basically assumes that a safe estimate for $R_k$ can be computed if
every higher-priority task $\tau_i$ is modelled as an ordinary sporadic
task with worst-case execution time $C_i$ and release jitter $S_i$.
Intuitively, it represents the potential internal jitter \textit{within} an activation of $\tau_i$, \ie, when its execution time $C_i$ is considered by disregarding any time intervals when $\tau_i$ is preempted. 
However, it is not the real jitter in the general cases, because the execution of $\tau_i$ can be pushed further, to be shown in the following example.


Consider the dynamic self-suspending task set presented in Table \ref{tab:counterexample-dynamic-suspension}. 
The analysis in Eq.~(\ref{eq:dynamic-flawed}) would yield $R_3=12$, as illustrated in 
Figure~\ref{fig:counterexample-dynamic}(a). However, the schedule of Figure~\ref{fig:counterexample-dynamic}(b), which is perfectly legal, 
disproves the claim that $R_3=12$, because $\tau_3$ in that case has a response time of $22-5\epsilon$ time units, 
where $\epsilon$ is an arbitrarily small quantity. 

\begin{table}[t]
\begin{center}
\begin{tabular}{|c|r|r|r|}
\hline
$\tau_i$ &      $C_i$   &   $S_i$  &     $T_i$     \\ \hline
$\tau_1$ &       $1$   &     $0$  &       $2$     \\ \hline
$\tau_2$  &      $5$   &     $5$  &      $20$     \\ \hline
$\tau_3$  &      $1$   &     $0$  &  $\infty$     \\ \hline
\end{tabular}
\end{center}
\caption{A set of dynamic self-suspending tasks for demonstrating the counterexample used for
  the incorrect quantification of jitter in Section \ref{sec:wrong-jitter-dynamic}.}
\label{tab:counterexample-dynamic-suspension}
\end{table}


\begin{figure}[t]
\centering
\def\uxfpga{0.4cm} 
\subfloat[An illustrative schedule based on Eq.~(\ref{eq:dynamic-flawed})]{
\scalebox{0.8}{
	\begin{tikzpicture}[x=\uxfpga,y=\uy,auto, thick]
    \draw[->] (-6,0) -- coordinate (xaxis) (18,0) node[anchor=north]{$t$};
    \foreach \x in {-6,-4,...,16}{
		\draw[-,below](\x,0) -- (\x,-0.3) node[] {\pgfmathtruncatemacro\yi{\x} \yi};
	}
	\foreach \x in {-6,-5,...,16}{
         \draw[-,very thin,lightgray, dashed](\x,0.3) -- (\x,6);
	}	
	\foreach \y in {2.03,4.03}{
		\draw[] (-6,\y) -- (16,\y);
	}
	
	\begin{scope}[shift={(0,0)}]
		\node[anchor=east] at (-6, 0.5) {$\tau_3$};
        \draw[->] (0,0) -- (0,1.75);
        \draw[<-,thin,red] (0,1.3) -- (5.3,1.3);
        \draw[->,thin,red] (6.7,1.3) -- (12,1.3);
        \draw[<-,thin,red] (0,1.3) -- (5,1.3);
        \draw[] (12.05,0) -- (12.05,1.5);
        \node[anchor=east,red] at (6.7, 1.39) {$12$};
        \draw[dotted] (12.5,0.5) -- (13.3,0.5); 
        \node[task7, minimum width=\uxfpga, anchor=south west] at (11, 0){\footnotesize};         
	\end{scope}

	\begin{scope}[shift={(0,2)}]
		\node[anchor=east] at (-6, 0.5) {$\tau_2$};
        \draw[->] (-5,0) -- (-5,1.75);
        \draw[->] (15,0) -- (15,1.75);
        \draw[dotted] (15.5,0.5) -- (16.3,0.5); 
		\foreach \y in {0.3,0.5,0.7}{ 
        		\draw[] (-5,\y) -- (0,\y);
        	} 
        \draw[] (0,0) -- (0,1);
        \foreach \x in {1,3,...,9}{ 
        		\node[task7, minimum width=\uxfpga, anchor=south west] at (\x, 0){\footnotesize};
        	}
	\end{scope}

	\begin{scope}[shift={(0,4)}]
		\node[anchor=east] at (-6, 0.5) {$\tau_1$};
        \foreach \x in {0,2,...,14}{ 
			\draw[->] (\x,0) -- (\x,1.75);        		
        		\node[task7, minimum width=\uxfpga, anchor=south west] at (\x, 0){\footnotesize};
        	}
        \draw[dotted] (15.5,0.5) -- (16.3,0.5); 
	\end{scope}                
\end{tikzpicture}} }

\subfloat[Another case with larger response time than that from the
schedule based on Eq.~(\ref{eq:dynamic-flawed})]{
\scalebox{0.8}{
	\begin{tikzpicture}[x=\uxfpga,y=\uy,auto, thick]
    \draw[->] (0,0) -- coordinate (xaxis) (36,0) node[anchor=north]{$t$};
    \foreach \x in {0,2,...,34}{
		\draw[-,below](\x,0) -- (\x,-0.3) node[] {\pgfmathtruncatemacro\yi{\x} \yi};
	}
	\foreach \x in {0,1,...,34}{
         \draw[-,very thin,lightgray, dashed](\x,0.3) -- (\x,6);
	}	
	\foreach \y in {2.03,4.03}{
		\draw[] (0,\y) -- (34,\y);
	}
	
	\begin{scope}[shift={(0,0)}]
		\node[anchor=east] at (0, 0.5) {$\tau_3$};
        \draw[->] (10,0) -- (10,1.75);
        \draw[dotted] (32.5,0.5) -- (33.3,0.5); 
        \draw[<-,thin,red] (10,1.3) -- (19,1.3);
        \draw[->,thin,red] (22.4,1.3) -- (31.55,1.3);
        \node[anchor=east,red] at (22.3, 1.39) {$22-5\varepsilon$};
        \node[task7, minimum width=0.1*\uxfpga, anchor=south east] at (20, 0){\footnotesize}; 
        \draw[] (31,1) -- (31.55,1);
        \draw[] (31,0) -- (31.55,0);
        \draw[] (31,0) -- (31,1);
        \draw[] (31.55,0) -- (31.55,1.5);   
	\end{scope}

	\begin{scope}[shift={(0,2)}]
		\node[anchor=east] at (0, 0.5) {$\tau_2$};
        \draw[->] (0,0) -- (0,1.75);
        \draw[->] (20,0) -- (20,1.75);
        \draw[dotted] (35.2,0.5) -- (36,0.5); 
        \draw[<-] (1.2,1) --(1.8,1.38);
        \node[anchor=east] at (2.7, 1.5) {$\varepsilon$};
        \draw[<-] (19.85,1) --(19.1,1.39);
        \node[anchor=east] at (19.2, 1.5) {$5\varepsilon$};        
		\foreach \y in {0.3,0.5,0.7}{ 
			\foreach \x in {1.2,3.2,...,9.2}{        		
        			\draw[] (\x,\y) -- (\x+0.8,\y);
        	}} 
        	\foreach \x in {1,3,...,9}{
        		\draw[] (\x,0) -- (\x,1);
        		\draw[] (\x+0.2,0) -- (\x+0.2,1);
        		\draw[] (\x+1,0) -- (\x+1,1);
        		\draw[] (\x,0) -- (\x+0.2,0);
        		\draw[] (\x,1) -- (\x+0.2,1);
        	}
    		\foreach \x in {11,13,...,17}{
			\node[task7, minimum width=\uxfpga, anchor=south west] at (\x, 0){\footnotesize};
		}	
    		\foreach \x in {21,23,...,29}{
			\node[task7, minimum width=\uxfpga, anchor=south west] at (\x, 0){\footnotesize};
		}
		\foreach \y in {0.3,0.5,0.7}{ 
    			\draw[] (30,\y) -- (30,\y);
        	}
		\foreach \y in {0.3,0.5,0.7}{ 
    			\draw[] (19.4,\y) -- (20,\y);
        	}
        	\draw[] (19,1) -- (19.4,1);
        \draw[] (19,0) -- (19.4,0);
        \draw[] (19,0) -- (19,1);
        \draw[] (19.4,0) -- (19.4,1);				
	\end{scope}

	\begin{scope}[shift={(0,4)}]
		\node[anchor=east] at (0, 0.5) {$\tau_1$};
		\foreach \x in {0,2,...,32}{ 
    			\draw[->] (\x,0) -- (\x,1.75);
    			\node[task7, minimum width=\uxfpga, anchor=south west] at (\x, 0){\footnotesize};
        	}        
        \draw[->] (34,0) -- (34,1.75);
        \draw[dotted] (34.5,0.5) -- (35.3,0.5); 
	\end{scope}
\end{tikzpicture}} }
\caption{A counterexample for the response time analysis based on
  Eq.~(\ref{eq:dynamic-flawed}) by using the task set in Table~\ref{tab:counterexample-dynamic-suspension}.}
\label{fig:counterexample-dynamic}
\end{figure}

%\begin{figure}[t]
%\begin{center}
%\includegraphics[width=\linewidth]{../figures/CounterexampleDynamicSuspension/counterexample_classic.png}
%\end{center}
%\caption{Two different schedules for the task set in Table~\ref{tab:counterexample-dynamic-suspension}.}
%\label{fig:counterexample-dynamic}
%\end{figure}

{\bf Consequences:} Since the results in \cite{ECRTS-AudsleyB04,RTAS-AudsleyB04,RTCSA-KimCPKH95,MingLiRTCSA1994} are fully based on the analysis in Eq.~(\ref{eq:dynamic-flawed}), the above unsafe example disproves the correctness of their analyses. The source of error comes from a wrong interpretation by Ming \cite{MingLiRTCSA1994} in 1994 with respect to a paper by Audsley et al. \cite{audsley-1993}.\footnote{The technical report of \cite{audsley-1993} was referred in \cite{MingLiRTCSA1994}. Here we refer to the journal version.} Audsley et al. \cite{audsley-1993} explained that deferrable executions may result in arrival jitter and the jitter terms should be accounted while analyzing the worst-case response time. However, Ming \cite{MingLiRTCSA1994} interpreted that the jitter is the self-suspension time, which was not originally provided in \cite{audsley-1993}. Therefore, there was no proof of the correctness of the methods used in \cite{MingLiRTCSA1994}. The concept was adopted by Kim et al. \cite{RTCSA-KimCPKH95} in 1995. 

This misconception spread further when it was propagated by Lakshmanan et al.~\cite{lakshmanan-2009} in their derivation of worst-case response time bounds for
partitioned multiprocessor real-time locking protocols, which in turn was reused in several later works~\cite{zeng-2011,bbb-2013,yang-2013,kim-2014,han-2014,carminati-2014,yang-2014}. We explain the consequences and how to correct the later analyses in \mysectionref{}~\ref{sec:syn}. 
 
Moreover this counterexample also invalidates the comparison in \cite{RidouardR06}, which compares the schedulability tests from \cite{RTCSA-KimCPKH95} and \cite[Page 164-165]{Liu:2000:RS:518501}, since the result derived from \cite{RTCSA-KimCPKH95} is unsafe.

Independently, the authors of the results in \cite{ECRTS-AudsleyB04,RTAS-AudsleyB04} used the same methods in 2004 from different perspectives. They already filed a technical report \cite{BletsasReport2015} to explain in a great detail how to handle this. 

{\bf Solutions:} It is explained and proved in \cite{huangpass:dac2015,BletsasReport2015} that the worst-case response time of task $\tau_k$ is the minimum $R_k$ with
\begin{equation}
R_k = C_k+ S_k+\sum_{\tau_i \in hp(k)}\ceiling{\frac{R_k+D_i-C_i}{T_i}} C_i,
\label{eq:dynamic-correct}
\end{equation}
for \emph{constrained-deadline} task systems under the assumption that every higher-priority task $\tau_i$ in $hp(k)$ can meet their relative deadline constraint. It is also safe to use $\ceiling{\frac{R_k+R_i-C_i}{T_i}}$ instead of $\ceiling{\frac{R_k+D_i-C_i}{T_i}}$ in the above equation if $R_i \leq D_i \leq T_i$.

\mysection{Incorrect Quantifications of Jitter (Segmented Self-Suspension)}
\label{sec:wrong-jitter-segmented}


We now explain the existing misconception in the literature to quantify the jitter too optimistically for segmented self-suspending task systems by using fixed-priority scheduling.  The analysis in \cite{RTCSA-BletsasA05} adopts two steps: 
\begin{enumerate}
\item The computation segments and the self-suspension intervals (including a ``notional''
self-suspension corresponding to the interval between the completion of the task and its next arrival)
are reordered such that the computation segments appear with decreasing execution time and
the suspension intervals appear with increasing self-suspending time.
\item Each computation segment is modelled as a sporadic task with a fixed offset corresponding to the above
rearrangement and a fixed jitter term to represent all computation segments of a given task.
As reported in \cite{RTCSA-BletsasA05}, this jitter term corresponds to the maximum internal jitter, within the 
activation of the task, of any computation segment, due to variability in the length of 
preceding computation segments and self-suspending intervals.
\end{enumerate}
The first step can be explained by using the following example of an
implicit-deadline segmented self-suspending task with $(C_i^1, S_i^1, C_i^2, S_i^2, C_i^3) = (1, 5, 4, 3, 2)$ and $T_i=40$.  It first artificially creates a notional gap $S_i^3=40-(1+5+4+3+2)=25$. After reordering, the task parameters become $(C_i^1, S_i^1, C_i^2, S_i^2, C_i^3, S_i^3)  = (4, 3, 2, 5, 1, 25)$.  The purpose of this reordering step was designed to avoid having to consider different release offsets for each interfering task (corresponding to its computational segments). 
%This is broadly analogous  to how Mok and Chen [REF] reorder task frames in their analysis for the multiframe task model.
The second step, which was designed to capture the effects of the variation in the length of 
computation segments or self-suspension intervals, would have no effect if 
there is no variation between the worst-case and the actual-case execution/suspension times.



Instead of going into the detailed mathematical formulations, we will demonstrate the misconception  in the above steps with the following example in Table~\ref{tab:counterexample-segmented} that has only one self-suspending task $\tau_3$ and there is no variation between the worst-case and the actual-case execution/suspension times.
In
this specific example,  neither step 1 nor step 2 has any effect. The
analysis in \cite{RTCSA-BletsasA05} can be imagined as replacing the
self-suspending task $\tau_3$ with a sporadic task without any jitter or self-suspension, with $C_3=2$ and $D_3=T_3=15$. Therefore, the analysis in \cite{RTCSA-BletsasA05}  concludes that the worst-case response time of task $\tau_4$ is at most $15$ since $C_4+\sum_{i=1}^{3}\ceiling{\frac{15}{T_i}} C_i = 3+ 6 + 4 + 2= 15$.


However, the perfectly legal schedule in Figure \ref{fig:counterexample-segmented} disproves this.
In that schedule, $\tau_1$, $\tau_2$, and $\tau_3$ arrive at $t=0$ and a job of $\tau_4$ arrives at $t=40$ and has a response time of 
$18$ time units.

\begin{table}[t]
\begin{center}
\begin{tabular}{|c||c|r|r|r|}
\hline
$\tau_i$ & $(C_i^1, S_i^1, C_i^2)$   &   $D_i$  &     $T_i$     \\ \hline
$\tau_1$ &  $(2, 0, 0)$                    &     $5$  &       $5$     \\ \hline
$\tau_2$ &  $(2, 0, 0)$                    &    $10$  &      $10$     \\ \hline
$\tau_3$ &  $(1, 5, 1)$            &    $15$  &      $15$     \\ \hline
$\tau_4$ &  $(3, 0, 0)$                   &    $?$  &   $\infty$    \\ \hline     
\end{tabular}
\end{center}
\caption{A set of segmented self-suspending tasks for demonstrating the misconception of the incorrect quantification of jitter in Section~\ref{sec:wrong-jitter-segmented}.}
\label{tab:counterexample-segmented}
\end{table}

\begin{figure}[t]
\centering
\def\uxfpga{0.25cm}
\scalebox{0.77}{
\begin{tikzpicture}[x=\uxfpga,y=\uy,auto, thick]
	\draw[->] (0,0) -- coordinate (xaxis) (62,0) node[anchor=north]{$t$};
	\foreach \x in {0,5,...,60}{
		\draw[-,below](\x,0) -- (\x,-0.3) node[] {\pgfmathtruncatemacro\yi{\x} \yi};
	}
	\foreach \x in {0,1,...,60}{
         \draw[-,very thin,lightgray, dashed](\x,0.3) -- (\x,8);
	}     
	\foreach \y in {2.03,4.03,6.03}{
		\draw[] (0,\y) -- (60,\y);
	}
       
	\begin{scope}[shift={(0,0)}]       
		\node[anchor=east] at (0, 0.5) {$\tau_4$};
		\draw[->] (40,0) -- (40,1.75);
		\draw[] (58.05,1) -- (58.05,1.5);
		\draw[dotted] (58.5,0.5) -- (60,0.5);
		\draw[<-,thin,red] (40,1.3) -- (48,1.3);
        	\draw[->,thin,red] (50,1.3) -- (58,1.3);
        	\node[anchor=east,red] at (50, 1.39) {18};
        	\node[task7, minimum width=2*\uxfpga, anchor=south west] at (48, 0){\footnotesize};         
        	\node[task7, minimum width=\uxfpga, anchor=south west] at (57, 0){\footnotesize};
	\end{scope}
      
    \begin{scope}[shift={(0,2)}]  
		\node[anchor=east] at (0, 0.5) {$\tau_3$};
		\foreach \x in {0,15,...,60}{
			\draw[<-](\x,1.75) -- (\x,0);
		}		
    		\draw[dotted] (60.5,0.5) -- (62,0.5);         
        	\node[task7, minimum width=\uxfpga, anchor=south west] at (4, 0){\footnotesize};
        	\foreach \y in {0.3,0.5,0.7}{ 
        		\draw[] (5,\y) -- (10,\y);
        	}
        	\draw[] (10,0) -- (10,1);
        	\node[task7, minimum width=\uxfpga, anchor=south west] at (14, 0){\footnotesize};
        	\node[task7, minimum width=\uxfpga, anchor=south west] at (17, 0){\footnotesize};
        	\foreach \y in {0.3,0.5,0.7}{ 
        	\draw[] (18,\y) -- (23,\y);}
        	\draw[] (23,0) -- (23,1);
        	\node[task7, minimum width=\uxfpga, anchor=south west] at (24, 0){\footnotesize};
        	\node[task7, minimum width=\uxfpga, anchor=south west] at (34, 0){\footnotesize};
        	\foreach \y in {0.3,0.5,0.7}{ 
        	\draw[] (35,\y) -- (40,\y);}
        	\draw[] (40,0) -- (40,1);
        	\node[task7, minimum width=\uxfpga, anchor=south west] at (44, 0){\footnotesize};
        	\node[task7, minimum width=\uxfpga, anchor=south west] at (47, 0){\footnotesize};
        	\foreach \y in {0.3,0.5,0.7}{ 
        		\draw[] (48,\y) -- (53,\y);
        	}
       	\draw[] (53,0) -- (53,1);
        	\node[task7, minimum width=\uxfpga, anchor=south west] at (54, 0){\footnotesize};
	\end{scope}
        
    \begin{scope}[shift={(0,4)}]        
		\node[anchor=east] at (0, 0.5) {$\tau_2$};
   		\foreach \x in {0,10,...,60}{
			\draw[<-](\x,1.75) -- (\x,0);
		}		
        	\draw[dotted] (60.5,0.5) -- (62,0.5);
        	\foreach \x in {2,12,...,52}{
			\node[task7, minimum width=2*\uxfpga, anchor=south west] at (\x, 0){\footnotesize};
		}
	\end{scope}

    \begin{scope}[shift={(0,6)}]        
		\node[anchor=east] at (0, 0.5) {$\tau_1$};
		\foreach \x in {0,5,...,55}{
			\draw[<-](\x,1.75) -- (\x,0);
			\node[task7, minimum width=2*\uxfpga, anchor=south west] at (\x, 0){\footnotesize};
		}
		\draw[<-](60,1.75) -- (60,0);		
		\draw[dotted] (60.5,0.5) -- (62,0.5);
	\end{scope}
\end{tikzpicture}}       
\caption{A schedule for demonstrating the misconception of the analysis in \cite{RTCSA-BletsasA05} by using the task set in Table  \ref{tab:counterexample-segmented}. }
\label{fig:counterexample-segmented}
\end{figure}

%\begin{figure}[t]
%\begin{center}
%\includegraphics[width=\linewidth]{../figures/CounterexampleSegmentedSuspension/counterexample_synthetic.png}
%\end{center}
%\caption{A schedule for the task set in Table  \ref{tab:counterexample-segmented}. }
%\label{fig:counterexample-segmented}
%\end{figure}

{\bf Consequences:} This example shows that the analysis in \cite{RTCSA-BletsasA05} is flawed.  The authors in \cite{RTCSA-BletsasA05}  already filed a technical report \cite{BletsasReport2015}.

{\bf Solutions:} When attempting to fix the error in the jitter quantification, there is no simple way to exploit the additional 
information provided by the segmented self-suspending task model.
However, quantifying the jitter of a self-suspending task $\tau_i$ with $D_i-C_i$ in Section~\ref{sec:wrong-jitter-dynamic} (or $R_i-C_i$) remains safe for constrained-deadline task systems since the dynamic self-suspending pattern is more general than a segmented self-suspending pattern.

\mysection{Incorrect Assumptions on the Critical Instant}
\label{sec:wrong-critical}

Over the years, it has been well accepted that the characterization of the critical instant for self-suspending tasks is a complex problem. The complexity of verifying the existence of a feasible schedule for segmented self-suspending tasks has been proven to be ${\cal NP}$-hard in the strong sense \cite{Ridouard_2004}.  For segmented self-suspending tasks with constrained deadlines under fixed-priority scheduling, 
the complexity of verifying the schedulability of a task set has been left open until a recent proof of its co${\cal NP}$-hardness in the strong sense by Chen \cite{RTSS2016-suspension} and Mohaqeqi et al. \cite{DBLP:conf/rtns/MohaqeqiE016} in 2016 (see \mysectionref{}~\ref{sec:hardness}).

Before that, Lakshmanan and Rajkumar \cite{LR:rtas10} proposed a worst-case response time analysis for a one-segmented self-suspending task $\tau_k$ (with one self-suspending interval) with pseudo-polynomial time complexity assuming that 
\begin{itemize}
\item the scheduling algorithm is fixed-priority;
\item $\tau_k$ is the lowest-priority task;  and
\item all the higher-priority tasks are sporadic and non-self-suspending.
\end{itemize}
The analysis, presented in \cite{LR:rtas10}, is based on the notion of
a critical instant, \ie, an instant at which, considering the state of the system, an execution request for $\tau_k$ will generate the largest response time. This critical instant was defined as follows:
\begin{itemize}
	\item every task releases a job simultaneously with $\tau_k$;
	\item the jobs of higher-priority tasks that are eligible to be released during the self-suspension interval of $\tau_k$ are delayed to be aligned with the release of the subsequent computation segment of $\tau_k$; and
	\item all the remaining jobs of the higher-priority tasks are released with their minimum inter-arrival time.
\end{itemize}

This definition of the critical instant is similar to the definition of the critical instant of a non-self-suspending task. Specifically, it is based on the two intuitions that $\tau_k$ suffers the worst-case interference when (i) all higher-priority tasks release their first jobs simultaneously with $\tau_k$ and (ii) they all release as many jobs as possible in each computation segment of $\tau_k$. Although intuitive, we provide examples that both statements are wrong. Note that the examples provided below already appeared in \cite{ecrts15nelissen}.

\mysubsection{A counterexample to the synchronous release}


\begin{figure}[t]
\centering
\def\uxfpga{0.4cm} 
\scalebox{1}{
	\begin{tikzpicture}[x=\uxfpga,y=\uy,auto, thick]
    \node[anchor=east] at (11.5, -1.75) {(a) Release jobs synchronously.};
    \node[anchor=east] at (28, -1.75) {(b) Do not release jobs synchronously.};
       
	\begin{scope}[shift={(0,0)}]       
    		\draw[->] (0,0) -- coordinate (xaxis) (12,0) node[anchor=north]{$t$};
    		\foreach \x in {0,2,...,10}{
			\draw[-,below](\x,0) -- (\x,-0.3) node[] {\pgfmathtruncatemacro\yi{\x} \yi};
		}
		\foreach \x in {0,1,...,10}{
        		\draw[-,very thin,lightgray, dashed](\x,0.3) -- (\x,6);
    		}
    		\foreach \y in {2.03,4.03}{
			\draw[] (0,\y) -- (10,\y);
		}		
		\node[anchor=east] at (0, 0.5) {$\tau_3$};
        \draw[->] (0,0) -- (0,1.75);
        \draw[dotted] (9.5,0.5) -- (10.3,0.5);
        \draw[<-,thin,red] (0,1.3) -- (3.9,1.3);
        \draw[->,thin,red] (5.1,1.3) -- (9,1.3);
        \draw[] (9.05,0) -- (9.05,1.5);
        \node[anchor=east,red] at (5, 1.39) {$9$}; 
        \node[task7, minimum width=\uxfpga, anchor=south west] at (2, 0){\footnotesize};     
        \node[task7, minimum width=3*\uxfpga, anchor=south west] at (6, 0){\footnotesize};    
		\foreach \y in {0.3,0.5,0.7}{        
			\draw[] (3,\y) -- (5,\y);
		}
		\draw[] (5,0) -- (5,1);
	\end{scope}
	
	\begin{scope}[shift={(0,2)}]       
		\node[anchor=east] at (0, 0.5) {$\tau_2$};
        \draw[->] (0,0) -- (0,1.75);
        \draw[dotted] (2.5,0.5) -- (3.3,0.5); 
        \node[task7, minimum width=\uxfpga, anchor=south west] at (1, 0){\footnotesize};
	\end{scope}	
        
    \begin{scope}[shift={(0,4)}]
    		\node[anchor=east] at (0, 0.5) {$\tau_1$};             
        \draw[->] (0,0) -- (0,1.75);
        \draw[dashed,->] (4,0) -- (4,1.75);
        \draw[->] (5,0) -- (5,1.75);
        \draw[->] (9,0) -- (9,1.75);
        \draw[dotted] (10.5,0.5) -- (11.3,0.5);
        \node[task7, minimum width=\uxfpga, anchor=south west] at (0, 0){\footnotesize};
        \node[task7, minimum width=\uxfpga, anchor=south west] at (5, 0){\footnotesize};
        \node[task7, minimum width=\uxfpga, anchor=south west] at (9, 0){\footnotesize};
	\end{scope}
	
	
	
	\begin{scope}[shift={(15,0)}]       
    		\draw[->] (0,0) -- coordinate (xaxis) (12,0) node[anchor=north]{$t$};
    		\foreach \x in {0,2,...,10}{
			\draw[-,below](\x,0) -- (\x,-0.3) node[] {\pgfmathtruncatemacro\yi{\x} \yi};
		}
		\foreach \x in {0,1,...,10}{
        		\draw[-,very thin,lightgray, dashed](\x,0.3) -- (\x,6);
    		}
    		\foreach \y in {2.03,4.03}{
			\draw[] (0,\y) -- (10,\y);
		}		
		\node[anchor=east] at (0, 0.5) {$\tau_3$};
        \draw[->] (0,0) -- (0,1.75);
        \draw[dotted] (10.5,0.5) -- (11.3,0.5); 
        \draw[<-,thin,red] (0,1.3) -- (4.3,1.3);
        \draw[->,thin,red] (5.7,1.3) -- (10,1.3);
        \draw[] (10.05,0) -- (10.05,1.5);
        \node[anchor=east,red] at (5.7, 1.39) {$10$}; 
        \node[task7, minimum width=\uxfpga, anchor=south west] at (1, 0){\footnotesize};     
        \node[task7, minimum width=2*\uxfpga, anchor=south west] at (6, 0){\footnotesize};
        \node[task7, minimum width=\uxfpga, anchor=south west] at (9, 0){\footnotesize};    
        \draw[] (4,0) -- (4,1);
		\foreach \y in {0.3,0.5,0.7}{        
			\draw[] (2,\y) -- (4,\y);
		}
	\end{scope}
	
	\begin{scope}[shift={(15,2)}]       
		\node[anchor=east] at (0, 0.5) {$\tau_2$};
        \draw[->] (4,0) -- (4,1.75);
        \draw[dotted] (6.5,0.5) -- (7.3,0.5); 
        \node[task7, minimum width=\uxfpga, anchor=south west] at (5, 0){\footnotesize};
	\end{scope}	
        
    \begin{scope}[shift={(15,4)}]             
        \node[anchor=east] at (0, 0.5) {$\tau_1$};
        \draw[dotted] (9.5,0.5) -- (10.3,0.5);
        \foreach \x in {0,4,8} {
			\draw[->] (\x,0) -- (\x,1.75);
			\node[task7, minimum width=\uxfpga, anchor=south west] at (\x, 0){\footnotesize};
        }
	\end{scope}	
\end{tikzpicture}}     
\caption{A counterexample to demonstrate the misconception of the synchronous release of all tasks in Section~\ref{sec:wrong-critical} based on the task set in Table~\ref{table:ex-synch-releases}.}
\label{fig:ex-synch-releases}
\end{figure}

\ifpaper
%\begin{figure}[t]
%  \centering
%\captionsetup[subfigure]{width=\columnwidth}
%  \subfloat[all tasks release a job synchronously.]{\label{fig:ex-phi} }\includegraphics[width=0.85\linewidth]{../figures/ex-phi/ex-phi.pdf} \\
%  \subfloat[all tasks do not release a job synchronously.]{\label{fig:ex-no-phi} }\includegraphics[width=0.85\linewidth]{../figures/ex-no-phi/ex-no-phi}
%  \caption{Counter-example to the synchronous release of all tasks (by \cite{LR:rtas10}).}
%  \label{fig:ex-synch-releases}
%\end{figure}
\fi

Consider three implicit deadline tasks with the parameters presented in Table~\ref{table:ex-synch-releases}. Let us assume that the priorities of the tasks are assigned using the rate monotonic policy (\ie, the smaller the period, the higher the priority). We are interested in computing the worst-case response time of $\tau_3$. Following the definition of the critical instant presented in \cite{LR:rtas10}, all three tasks must release a job synchronously at time $0$. Using the standard response-time analysis for non-self-suspending tasks, we get that the worst-case response time of the first computation segment of $\tau_3$ is equal to $R_3^1 = 3$. Because the second job of $\tau_1$ would be released in the self-suspending interval of $\tau_3$ if $\tau_1$ was strictly respecting its minimum inter-arrival time, the release of the second job of $\tau_1$ is delayed so as to coincide with the release of the second computation segment of $\tau_3$ (see Figure~\ref{fig:ex-synch-releases}(a)). Considering the fact that the second job of $\tau_2$ cannot be released before time instant $50$ and hence does not interfere with the execution of $\tau_3$, the response time of the second computation segment of $\tau_3$ is thus equal to $R_3^2=4$. In total, the worst-case response time of $\tau_3$ when all tasks release a job synchronously is equal to 
$$R_3 = R_3^1 + S_3^1 + R_3^2 = 3 + 2 +4 = 9$$

\begin{table}[t] 
\centering
    \begin{tabular}{|c|c|c|}
 \hline
        & $(C_i^1, S_i^1, C_i^2)$ &  $D_i=T_i$\\ 
        \hline
        $\tau_1$ & (1, 0, 0) &  4\\ 
        $\tau_2$ &  (1, 0, 0) & 50  \\ 
        $\tau_3$ & (1, 2, 3) & 100  \\
        \hline
    \end{tabular} 
    \caption{A set of segmented self-suspending tasks for demonstrating the misconception
of the synchronous release of all tasks in Section~\ref{sec:wrong-critical}.}
    \label{table:ex-synch-releases}
\end{table}

Now, consider a job release pattern as shown in Figure~\ref{fig:ex-synch-releases}(b). Task $\tau_2$ does not release a job synchronously with task $\tau_3$ but with its second computation segment instead. The response time of the first computation segment of $\tau_3$ is thus reduced to $R_3^1=2$. However, both $\tau_1$ and $\tau_2$ can now release a job synchronously with the second computation segment of $\tau_3$, for which the response time is now equal to $R_3^2=6$ (see Figure~\ref{fig:ex-synch-releases}(b)). Thus, the total response time of $\tau_3$ in a scenario where not all higher-priority tasks release a job synchronously with $\tau_3$ is equal to 
$$R_3 = R_3^1 + S_3^1 + R_3^2 = 2+2+6 = 10$$

{\bf Consequence:}  The synchronous release of all tasks does not necessarily generate the maximum interference for the self-suspending task $\tau_k$ and is thus not always a critical instant for $\tau_k$. 
It was however proven in \cite{ecrts15nelissen} that in the critical instant of a self-suspending task $\tau_k$, every higher-priority task releases a job synchronously with the arrival of at least one computation segment of $\tau_k$, but not all higher-priority tasks must release a job synchronously with the same computation segment.

\mysubsection{A counterexample to the minimum inter-release time}

Consider a task set of $4$ tasks $\tau_1, \tau_2, \tau_3, \tau_4$ in which $\tau_1$, $\tau_2$ and $\tau_3$ are non-self-suspending sporadic tasks and $\tau_4$ is a self-suspending task with the lowest priority. The tasks have the parameters provided in Table~\ref{table:ex-num-releases}. The worst-case response time of $\tau_4$ is obtained when $\tau_1$ releases a job synchronously with the second computation segment of $\tau_4$ while $\tau_2$ and $\tau_3$ must release a job synchronously with the first computation segment of $\tau_4$.

\begin{table}[t]
\centering
    \begin{tabular}{|c|c|c|}
 \hline
        & $(C_i^1, S_i^2, C_i^2)$ &  $D_i=T_i$\\ 
        \hline
        $\tau_1$ & (4, 0, 0) &  8\\ 
        $\tau_2$ &  (1, 0, 0) & 10 \\ 
        $\tau_3$ & (1, 0, 0) & 17 \\
        $\tau_4$ & (265, 2, 6) & 1000\\
        \hline
    \end{tabular} 
    \caption{A set of segmented self-suspending tasks for demonstrating the misconception to release the jobs as early and often as possible for interfering with the computation segments of task $\tau_k$ in Section~\ref{sec:wrong-critical}. }
    \label{table:ex-num-releases}
\end{table}

Consider two scenarios with respect to the job release pattern. Scenario~1 is a result of the proposed critical instant, in which the jobs of the higher-priority non-self-suspending tasks are released as early and often as possible in each computation segment of $\tau_4$. We show that the WCRT of $\tau_4$ in another scenario, i.e., Scenario~2, is higher.
 In Scenario~2, one less job of task $\tau_1$ is released in (and therefore interferes with) the first computation segment of the self-suspending task.

Scenario~1 is depicted in Fig.~\ref{fig:ex_crit_inst2}(a), and Scenario~2 in Fig.~\ref{fig:ex_crit_inst2}(b). The first $765$ time units are omitted in both figures. This is mainly due to space constraint. In both scenarios, the schedules of the jobs are identical in this initial time window.  
The first jobs of $\tau_1$, $\tau_2$, and $\tau_3$ are released synchronously with the arrival of the first computation segment of $\tau_4$ at time $0$. The subsequent jobs of these three tasks are released as early and often as possible respecting the minimum inter-arrival times of the respective tasks. That is, they are released periodically with periods $T_1$, $T_2$ and $T_3$, respectively. With this release pattern, it is easy to compute that the $97^\text{th}$ job of $\tau_1$ is released at time $768$, the $78^\text{th}$ job of $\tau_2$ at time $770$ and the $46^\text{th}$ job of $\tau_3$ at time $765$. As a consequence, at time $765$, $\tau_4$ has finished executing $259$ time units of its first execution segment out of $265$ in both scenarios, \ie, $765 - 96 \times 4 - 77 \times 1 - 45 \times 1 = 259$.  From time $765$ onward, we separately consider Scenarios~1 and~2.
%\begin{figure}
%  \centering
%  \subfloat[Scenario 1. Jobs are released as often as possible.]{\label{fig:ex_crit_inst2sc1} \includegraphics[height=1.95cm, width=\linewidth]{../figures/ex2sc1}} \\
%  \subfloat[Scenario 2. Jobs are not released as often as possible.]{\label{fig:ex_crit_inst2sc2} %\includegraphics[height=1.8cm, width=\linewidth]{../figures/ex2sc2}}
%  \caption{Example showing that releasing higher priority jobs as often as possible may not always cause the maximum interference on a self-suspending task.}
%  \label{fig:ex_crit_inst2}
%\end{figure}

\begin{figure}[t]
\centering
\def\uxfpga{0.4cm} 
\subfloat[Scenario 1. Jobs are released as early and often as possible to interfere with each computation segment of task $\tau_k$.]{
\scalebox{0.73}{
	\begin{tikzpicture}[x=\uxfpga,y=\uy,auto, thick]
    \draw[->] (0,0) -- coordinate (xaxis) (39,0) node[anchor=north]{$t$};
    \foreach \x in {0,5,...,35}{
		\draw[-,below](\x,0) -- (\x,-0.3) node[] {\pgfmathtruncatemacro\yi{\x+765} \yi};
	}
	\foreach \x in {0,1,...,38}{
         \draw[-,very thin,lightgray, dashed](\x,0) -- (\x,8);
	}	
	\foreach \y in {2.03,4.03,6.03}{
		\draw[] (0,\y) -- (38,\y);
	}
	
	\begin{scope}[shift={(0,0)}]
		\node[anchor=east] at (0, 0.5) {$\tau_4$};
        \node[task7, minimum width=2*\uxfpga, anchor=south west] at (1, 0){\footnotesize};
        \node[task7, minimum width=3*\uxfpga, anchor=south west] at (8, 0){\footnotesize};
        \node[task7, minimum width=\uxfpga, anchor=south west] at (16, 0){\footnotesize};
        \node[task7, minimum width=\uxfpga, anchor=south west] at (24, 0){\footnotesize};
        \node[task7, minimum width=\uxfpga, anchor=south west] at (26, 0){\footnotesize};
        \node[task7, minimum width=4*\uxfpga, anchor=south west] at (31, 0){\footnotesize};  
    		\foreach \y in {0.3,0.5,0.7}{ 
    			\draw[] (17,\y) -- (19,\y);
        	} 
        \draw[] (19,0) -- (19,1);
	\end{scope}

	\begin{scope}[shift={(0,2)}]
		\node[anchor=east] at (0, 0.5) {$\tau_3$};
        \draw[->] (0,0) -- (0,1.75);
        \draw[->] (19,0) -- (19,1.75);
        \draw[->] (36,0) -- (36,1.75);
        \draw[->, dashed] (17,0) -- (17,1.75);
        \draw[<->, thin, red] (17,0.5)--(19,0.5);
        \node[anchor=east,red] at (19.3, 1) {Delay}; 
    		\node[task7, minimum width=\uxfpga, anchor=south west] at (0, 0){\footnotesize};
    		\node[task7, minimum width=\uxfpga, anchor=south west] at (23, 0){\footnotesize};
	\end{scope}

	\begin{scope}[shift={(0,4)}]
		\node[anchor=east] at (0, 0.5) {$\tau_2$};
        \foreach \x in {5,15,...,35}{ 
			\draw[->] (\x,0) -- (\x,1.75);            		
        	}
        	\node[task7, minimum width=\uxfpga, anchor=south west] at (7, 0){\footnotesize};
        	\node[task7, minimum width=\uxfpga, anchor=south west] at (15, 0){\footnotesize};
        	\node[task7, minimum width=\uxfpga, anchor=south west] at (25, 0){\footnotesize};
	\end{scope}                

	\begin{scope}[shift={(0,6)}]
		\node[anchor=east] at (0, 0.5) {$\tau_1$};
        \foreach \x in {3,11,...,27}{ 
			\draw[->] (\x,0) -- (\x,1.75);     
			\node[task7, minimum width=4*\uxfpga, anchor=south west] at (\x, 0){\footnotesize};       		
        	}
        	\draw[->] (35,0) -- (35,1.75);
	\end{scope}                
	
\end{tikzpicture}} }

\subfloat[Scenario 2. Jobs are not released as early and often as possible.]{
\scalebox{0.73}{
	\begin{tikzpicture}[x=\uxfpga,y=\uy,auto, thick]
    \draw[->] (0,0) -- coordinate (xaxis) (39,0) node[anchor=north]{$t$};
    \foreach \x in {0,5,...,35}{
		\draw[-,below](\x,0) -- (\x,-0.3) node[] {\pgfmathtruncatemacro\yi{\x+765} \yi};
	}
	\foreach \x in {0,1,...,38}{
         \draw[-,very thin,lightgray, dashed](\x,0) -- (\x,8);
	}	
	\foreach \y in {2.03,4.03,6.03}{
		\draw[] (0,\y) -- (38,\y);
	}
	
	\begin{scope}[shift={(0,0)}]
		\node[anchor=east] at (0, 0.5) {$\tau_4$};
        \node[task7, minimum width=2*\uxfpga, anchor=south west] at (1, 0){\footnotesize};
        \node[task7, minimum width=4*\uxfpga, anchor=south west] at (8, 0){\footnotesize};
        \node[task7, minimum width=2*\uxfpga, anchor=south west] at (20, 0){\footnotesize};
        \node[task7, minimum width=3*\uxfpga, anchor=south west] at (27, 0){\footnotesize};
        \node[task7, minimum width=\uxfpga, anchor=south west] at (36, 0){\footnotesize};
    		\foreach \y in {0.3,0.5,0.7}{ 
    			\draw[] (12,\y) -- (14,\y);
        	} 
        \draw[] (14,0) -- (14,1);
	\end{scope}

	\begin{scope}[shift={(0,2)}]
		\node[anchor=east] at (0, 0.5) {$\tau_3$};
        \draw[->] (0,0) -- (0,1.75);
        \draw[->] (17,0) -- (17,1.75);
        \draw[->] (34,0) -- (34,1.75);
    		\node[task7, minimum width=\uxfpga, anchor=south west] at (0, 0){\footnotesize};
    		\node[task7, minimum width=\uxfpga, anchor=south west] at (19, 0){\footnotesize};
    		\node[task7, minimum width=\uxfpga, anchor=south west] at (34, 0){\footnotesize};
	\end{scope}

	\begin{scope}[shift={(0,4)}]
		\node[anchor=east] at (0, 0.5) {$\tau_2$};
        \foreach \x in {5,15,...,35}{ 
			\draw[->] (\x,0) -- (\x,1.75);            		
        	}
        	\node[task7, minimum width=\uxfpga, anchor=south west] at (7, 0){\footnotesize};
        	\node[task7, minimum width=\uxfpga, anchor=south west] at (18, 0){\footnotesize};
        	\node[task7, minimum width=\uxfpga, anchor=south west] at (26, 0){\footnotesize};
        	\node[task7, minimum width=\uxfpga, anchor=south west] at (35, 0){\footnotesize};
	\end{scope}                

	\begin{scope}[shift={(0,6)}]
		\node[anchor=east] at (0, 0.5) {$\tau_1$};
        \foreach \x in {3,14,22,30}{ 
			\draw[->] (\x,0) -- (\x,1.75);     
			\node[task7, minimum width=4*\uxfpga, anchor=south west] at (\x, 0){\footnotesize};       		
        	}
        	\draw[->] (38,0) -- (38,1.75);
        \draw[->, dashed] (11,0) -- (11,1.75);
        \draw[<->, thin, red] (11,0.5)--(14,0.5);
        \node[anchor=east,red] at (13.8, 1) {Delay}; 

	\end{scope}                
\end{tikzpicture}} }
\caption{An example based on the task set in Table~\ref{table:ex-num-releases} showing that releasing higher-priority jobs as early and often as possible to interfere with each computation segment of task $\tau_k$ may not always cause the maximum interference on a self-suspending task.}
\label{fig:ex_crit_inst2}
\end{figure}

\noindent\textbf{Scenario 1.} Continuing the release of jobs of the non-self-suspending tasks as early and often as possible without violating their minimum inter-arrival times, the first computation segment of $\tau_4$ finishes its execution at time $782$ as shown in Fig.~\ref{fig:ex_crit_inst2}(a). After completion of its first computation segment, $\tau_4$ self-suspends for two time units until time $784$. As $\tau_3$ would have released a job within the self-suspending interval, we delay the release of that job from time $782$ to $784$ in order to maximize the interference exerted by $\tau_3$ on the second computation segment of $\tau_4$ as shown in Fig.~\ref{fig:ex_crit_inst2}(a). Note that, in order to respect its minimum inter-arrival time, $\tau_2$ has an offset of $6$ time units with the arrival of the second computation segment of $\tau_4$. Upon following the rest of the schedule, it can easily be seen that the job of $\tau_4$ finishes its execution at time $800$.

\noindent\textbf{Scenario 2.} As shown in Fig.~\ref{fig:ex_crit_inst2}(b), the release of a job of task $\tau_1$ is skipped at time $776$ in comparison to Scenario~1. As a result, the execution of the first computation segment of $\tau_4$ is completed at time $777$, thereby causing one job of $\tau_2$ that was released at time $780$ in Scenario 1, to \emph{not} be released during the execution of the first computation segment of $\tau_4$. The response time of the first computation segment of $\tau_4$ is thus reduced by $C_1 + C_2 = 5$ time units in comparison to Scenario 1 (see Fig.~\ref{fig:ex_crit_inst2}(a)). Note that this deviation from Scenario 1 does not affect the fact that $\tau_1$ still releases a job synchronously with the second computation segment of $\tau_4$. The next job of $\tau_3$ however, is not released in the suspension interval anymore but $3$ time units after the arrival of $\tau_4$'s second computation segment. Moreover, the offset of $\tau_2$ with respect to the start of the second computation segment is reduced by $C_1 + C_2 = 5$ time units. This causes an extra job of $\tau_2$ to be released in the second computation segment of $\tau_4$, initiating a cascade effect: an extra job of $\tau_1$ is released in the second computation segment at time $795$, which in turn causes the release of an extra job of $\tau_3$, itself causing the arrival of one more job of $\tau_2$. Consequently, the response time of the second computation segment increases by $C_2 + C_1 + C_3 + C_2 = 7$ time units. Overall, the response time of $\tau_4$ increases by $7 - 5 = 2$ time units in comparison to Scenario~1. This is reflected in Figure~\ref{fig:ex_crit_inst2}(b) as the job of $\tau_4$ finishes its execution at time $802$.\\


{\bf Consequence:} This counterexample proves that the response time of a self-suspending task $\tau_k$ can be larger when the tasks in $hp(k)$ do not release jobs as early and often as possible to interfere with each computation segment of task $\tau_k$.

{\bf Solution:} The problem of defining the critical instant remains open even for the special case where only the lowest-priority task is self-suspending. Nelissen et al. propose a limited solution in \cite{ecrts15nelissen} based on an exhaustive search with exponential time complexity.

\mysection{Counting Highest-Priority Self-Suspending Time to Reduce the Interference}
\label{sec:wrong-highest-priority}

We now present a misconception which exploits the  self-suspension time  of the highest-priority task to reduce its interference to the lower-priority sporadic tasks. 
We consider fixed-priority preemptive scheduling for $n$ self-suspending sporadic real-time tasks on a single processor, in which $\tau_1$ is the highest-priority task and $\tau_n$ is the lowest-priority task.
% We focus on constrained-deadline task systems with $D_i \leq T_i$ or implicit-deadline systems with $D_i=T_i$ for $i=1,\ldots,n$.
Let us consider the simplest setting of such a case:
\begin{itemize}
\item there is only one self-suspending task with the highest priority, \ie, $\tau_1$,
\item the self-suspending time is fixed, \ie, early return of self-suspension has to be controlled by the scheduler, and
\item the actual execution time of the self-suspending task is always equal to its worst-case execution time.
\end{itemize}
Denote this task set as $\Gamma_{1s}$ (as also used in \cite{RTSS-KimANR13}).  Since $\tau_1$ is the highest-priority task, its execution behavior is static under the above assumptions. The misconception here is to identify the critical instant  (Theorem 2 in \cite{RTSS-KimANR13}) as follows: ``a critical instant occurs when all the tasks are released at the same time if $C_1 +S_1 < C_i  \leq T_1-C_1-S_1 \mbox{ for } i \in\{i|i\in Z^{+} \mbox{ and } 1<i\leq n\}$ is satisfied.'' This observation leads to the wrong implication which uses the self-suspension time (if it is long enough) to \emph{reduce} the computation demand of $\tau_i$ for interfering with lower-priority tasks. 


\begin{table} [t]
\centering
    \begin{tabular}{|c|c|c|}
 \hline
        & $(C_i^1, S_i^1, C_i^2)$ &  $D_i=T_i$\\ 
        \hline
        $\tau_1$ & $(\epsilon, 1, 1)$ &  $4+10\epsilon$\\ 
        $\tau_2$ &  ($2+2\epsilon$, 0, 0) & 6  \\ 
        $\tau_3$ & ($2+2\epsilon$, 0, 0) & 6  \\
        \hline
    \end{tabular} 
    \caption{A set of segmented self-suspending tasks for demonstrating the misconception
to reduce the interference by exploiting the highest-priority self-suspending time in Section~\ref{sec:wrong-highest-priority}, where $0 < \epsilon \leq 0.1$.}
    \label{table:ex-highest-priority}
\end{table}


{\it Counterexample to Theorem 2 in \cite{RTSS-KimANR13}:} Let $\epsilon$ be a positive and very small number, \ie, $0 < \epsilon \leq 0.1$.  Consider the three tasks listed in Table~\ref{table:ex-highest-priority}. By the setting, $2+\epsilon = C_1+S_1 < C_i = 2+2\epsilon \leq T_1-C_1-S_1 = 2+9\epsilon$ for $i=2,3$. The above theorem states that the worst case is to release all the three tasks together at time $0$ (as shown in Figure~\ref{fig:counterexample-reduce-interf}(a)). The analysis shows that the response time of task $\tau_3$ is at most $5+6\epsilon$. However, if we release task $\tau_1$ at time $0$ and release task $\tau_2$ and task $\tau_3$ at time $1+\epsilon$ (as shown in Figure~\ref{fig:counterexample-reduce-interf}(b)), the response time of the first job of task $\tau_3$ is $6+5\epsilon$. 

\begin{figure}[t]
\centering
\def\uxfpga{0.4cm} 
\scalebox{1}{
	\begin{tikzpicture}[x=\uxfpga,y=\uy,auto, thick]
    \node[anchor=east] at (11.5, -1.75) {(a) Release jobs synchronously.};
    \node[anchor=east] at (28, -1.75) {(b) Do not release jobs synchronously.};
       
	\begin{scope}[shift={(0,0)}]       
    		\draw[->] (0,0) -- coordinate (xaxis) (12,0) node[anchor=north]{$t$};
	    \foreach \x in {0,2,...,10}{
			\draw[-,below](\x,0) -- (\x,-0.3) node[] {\pgfmathtruncatemacro\yi{\x} \yi};
		}
		\foreach \x in {0,1,...,10}{
        		\draw[-,very thin,lightgray, dashed](\x,0.3) -- (\x,6);
    		}
    		\foreach \y in {2.03,4.03}{
			\draw[] (0,\y) -- (10,\y);
		}
		\node[anchor=east] at (0, 0.5) {$\tau_3$};
        \draw[->] (0,0) -- (0,1.75);
        \draw[->] (6,0) -- (6,1.75);       
        \draw[dotted] (6.5,0.5) -- (7.3,0.5);
        \draw[<-,thin,red] (0,1.3) -- (1.5,1.3);
        \draw[->,thin,red] (4.1,1.3) -- (5.6,1.3);
        \node[anchor=east,red] at (4.2, 1.39) {$5+6\varepsilon$};         
        \draw[] (3.3,1.03) -- (4.8,1.03);
        \draw[] (3.3,0) -- (3.3,1);
        \draw[] (4.8,0) -- (4.8,1);
        \draw[] (5,1.03) -- (5.6,1.03);
        \draw[] (5,0) -- (5,1);
        \draw[] (5.6,0) -- (5.6,1.5);
	\end{scope}
	
	\begin{scope}[shift={(0,2)}]       
        \node[anchor=east] at (0, 0.5) {$\tau_2$};
        \draw[->] (0,0) -- (0,1.75);
        \draw[->] (6,0) -- (6,1.75);
        \draw[dotted] (6.5,0.5) -- (7.3,0.5); 
        \node[task7, minimum width=\uxfpga, anchor=south west] at (0.2, 0){\footnotesize};
        \draw[] (2.2,0.03) -- (3.3,0.03);
        \draw[] (2.2,1.03) -- (3.3,1.03);
        \draw[] (2.2,0.03) -- (2.2,1.03);
        \draw[] (3.3,0.03) -- (3.3,1.03);
	\end{scope}	
        
    \begin{scope}[shift={(0,4)}]
    		\node[anchor=east] at (0, 0.5) {$\tau_1$};
    		\draw[dotted] (10.1,0.5) -- (11,0.5);    		
    		\foreach \x in {0,4.8}{
    			\draw[->] (\x,0) -- (\x,1.75);
    			\node[task7, minimum width=\uxfpga, anchor=south west] at (\x+1.2, 0){\footnotesize};
    			\draw[] (\x,1.03) -- (\x+0.2,1.03);
    			\draw[] (\x,0.03) -- (\x+0.2,0.03);
    			\draw[] (\x+0.2,0) -- (\x+0.2,1);
    			\foreach \y in {0.3,0.5,0.7}{
    			\draw[] (\x+0.2,\y) -- (\x+1.2,\y);
    		}}
    		\draw[->] (9.6,0) -- (9.6,1.75);
	\end{scope}
	
	
	
	
	\begin{scope}[shift={(15,0)}]       
    		\draw[->] (0,0) -- coordinate (xaxis) (12,0) node[anchor=north]{$t$};
	    \foreach \x in {0,2,...,10}{
			\draw[-,below](\x,0) -- (\x,-0.3) node[] {\pgfmathtruncatemacro\yi{\x} \yi};
		}
		\foreach \x in {0,1,...,10}{
        		\draw[-,very thin,lightgray, dashed](\x,0.3) -- (\x,6);
    		}
    		\foreach \y in {2.03,4.03}{
			\draw[] (0,\y) -- (10,\y);
		}
        \node[anchor=east] at (0, 0.5) {$\tau_3$};
        \draw[->] (1.2,0) -- (1.2,1.75);
        \draw[<-,red] (7.2,0) -- (8.6,1.3);
        \node[anchor=east,red] at (10.5, 1.68) {miss};
        \draw[<-,thin,red] (1.2,1.3) -- (3,1.3);
        \draw[->,thin,red] (5.7,1.3) -- (7.6,1.3);
        \node[anchor=east,red] at (5.8, 1.4) {$6+5\varepsilon$};
        \draw[] (4.25,1.03) -- (4.8,1.03);
        \draw[] (4.25,0.03) -- (4.8,0.03);
        \draw[] (4.25,0.03) -- (4.25,1.03);
        \draw[] (4.8,0.03) -- (4.8,1.03);
        \node[task7, minimum width=\uxfpga, anchor=south west] at (5, 0){\footnotesize};
        \draw[] (7,1.03) -- (7.6,1.03);
        \draw[] (7,0.03) -- (7,1.03);
        \draw[] (7.6,0.03) -- (7.6,1.5);
	\end{scope}
	
	\begin{scope}[shift={(15,2)}]       
		\node[anchor=east] at (0, 0.5) {$\tau_2$};
        \draw[->] (1.2,0) -- (1.2,1.75);
        \draw[->] (7.2,0) -- (7.2,1.75);
        \draw[] (2.2,1.03) -- (4.25,1.03);
        \draw[] (2.2,0.03) -- (4.25,0.03);
        \draw[] (2.2,0.03) -- (2.2,1.03);
        \draw[] (4.25,0.03) -- (4.25,1.03);
	\end{scope}	
        
    \begin{scope}[shift={(15,4)}]             
    		\node[anchor=east] at (0, 0.5) {$\tau_1$};
    		\foreach \x in {0,4.8}{
    			\draw[->] (\x,0) -- (\x,1.75);
    			\node[task7, minimum width=\uxfpga, anchor=south west] at (\x+1.2, 0){\footnotesize};
    			\draw[] (\x,1.03) -- (\x+0.2,1.03);
    			\draw[] (\x,0.03) -- (\x+0.2,0.03);
    			\draw[] (\x+0.2,0) -- (\x+0.2,1);
    			\foreach \y in {0.3,0.5,0.7}{
    			\draw[] (\x+0.2,\y) -- (\x+1.2,\y);
    		}}
    		\draw[->] (9.6,0) -- (9.6,1.75);
	\end{scope}	
 \end{tikzpicture}}     
  \caption{A counterexample presented in Section~\ref{sec:wrong-highest-priority} for demonstrating the misconception on the synchronous release used in Theorem 2 in \cite{RTSS-KimANR13}, based on the task set in Table~\ref{table:ex-highest-priority}.}
  \label{fig:counterexample-reduce-interf}
\end{figure}

This misconception also leads to a wrong statement in Theorem 3 in \cite{RTSS-KimANR13}:
\begin{quote}
{\it Theorem 3 in \cite{RTSS-KimANR13}}: For a taskset $\Gamma_{1s}$ with implicit deadlines, $\Gamma_{1s}$ is schedulable if the total utilization of the taskset is less than or equal to $n((2+2\gamma)^{\frac{1}{n}}-1)-\gamma$, where $n$ is the number of tasks in $\Gamma_{1s}$, and $\gamma$ is the ratio of
$S_1$ to $T_1$ and lies in the range of $0$ to $2^{\frac{1}{n-1}}-1$. 
\end{quote}


{\it Counterexample of Theorem 3 in \cite{RTSS-KimANR13}:} Suppose that the self-suspending task $\tau_1$ has two computation segments, with $C_1^1 = C_1-\epsilon$, $C_1^2 = \epsilon$, and $S_1=S_1^1 > 0$ with very small $0 < \epsilon \ll C_1^1$. For such an example, it is pretty obvious that this self-suspending highest-priority task is like an ordinary sporadic task, \ie, self-suspension does not matter. 
 In this counterexample, the utilization bound is still Liu and Layland bound $n(2^{\frac{1}{n}}-1)$ \cite{Liu_1973}, regardless of the ratio of $S_1/T_1$. 

The source of the error of Theorem 3 in \cite{RTSS-KimANR13} is due to its Theorem 2 and the footnote 4 in \cite{RTSS-KimANR13}, which claims that the case in Figure 7 in \cite{RTSS-KimANR13} is the worst case. This statement is incorrect and can be disproved with the above counterexample. 


{\bf Consequences:} Theorems 2 and 3 in \cite{RTSS-KimANR13} are flawed.  

{\bf Solutions:} The three assumptions, \ie, one highest-priority segmented self-suspending task, controlled suspension behavior and controlled execution time  in \cite{RTSS-KimANR13} actually imply that the self-suspending behavior of task $\tau_1$ can be modeled as several sporadic tasks with the same minimum inter-arrival time. That is, if the $j$-th computation segment of task $\tau_1$ starts its execution at time $t$, the earliest time for this computation segment to be executed again in the next job of task $\tau_1$ is at least $t+T_1$. Therefore, a constrained-deadline task $\tau_k$ can be feasibly scheduled by the fixed-priority scheduling strategy if $C_1+S_1 \leq D_1$ and for $2 \leq k \leq n$
  \begin{equation}
    \label{eq:tda-fixed}
\exists 0 < t \leq D_k, \qquad C_k + \sum_{i=1}^{k-1}\ceiling{\frac{t}{T_i}}C_i \leq t.    
  \end{equation}
%This also implies that the utilization bound (for implicit-deadline task systems) is still Liu and Layland bound $n(2^{\frac{1}{n}}-1)$ \cite{Liu_1973}, regardless of the ratio of $S_1/T_1$. 

A version of~\cite{RTSS-KimANR13} correcting the problems mentioned in this section can be found in~\cite{Kim2016}.

\mysection{Incorrect Segmented Fixed-Priority Scheduling with Periodic Enforcement}
\label{sec:wrong-periodic}

We now introduce misconceptions that may happen due to periodic enforcement if it is not carefully adopted for segmented self-suspending task systems. As mentioned in Section~\ref{sec:static-period-enforce}, we can set a constant offset to constrain the release time of a computation segment. If this offset is given, each computation segment behaves like a standard sporadic (or periodic) task. Therefore, the schedulability test for sporadic task systems can be directly applied. Since the offsets of two computation segments of a task may be different, one may want to assign each computation segment a \emph{fixed-priority} level.  However, this has to be carefully handled. 



Consider the example listed in Table~\ref{table:ex-periodic}. Suppose that the offset of the computation segment $C_2^1$ is $0$ and the offset of the computation segment $C_2^2$ is $10$. This setting creates three sporadic tasks.
Suppose that the segmented fixed priority assignment assigns $C_2^1$ the highest priority and $C_2^2$ the lowest priority. It should be clear that the worst-case response time of $C_2^1$ is $5$ and the worst-case response time of $C_1$ is $15$. We focus on the WCRT analysis of $C_2^2$.


Since the two computation segments of task $\tau_2$ should not have any overlap, one may think that during the analysis of the worst-case response time of $C_2^2$, we do not have to consider the computation segment $C_2^1$. The worst-case response time of $C_2^2$ (after its constant offset $10$) for this case is $26$ since $\ceiling{\frac{26}{30}} C_1 + C_2^2 = 26$. 
Since $26+10 < 40$, one may conclude that this enforcement results in a feasible schedule. This analysis is adopted in Section IV in \cite{RTSS-KimANR13} and Section 3 in \cite{DBLP:journals/ieicet/DingTT09}. 

Unfortunately, this analysis is incorrect.
%it is not.
Figure~\ref{fig:counterexample-FP-segment-level} provides a concrete schedule, in which the response time of $C_2^2$ is larger than $30$, which leads to a deadline miss.
% In fact, the $5$ units of execution time of $C_2^1$ push $C_1$ and result in a deadline miss of task $\tau_2$.

\begin{table} [t]
\centering
    \begin{tabular}{|c|c|c|}
 \hline
        & $(C_i^1, S_i^1, C_i^2)$ &  $D_i=T_i$\\ 
        \hline
        $\tau_1$ & $(10, 0, 0)$ &  $30$\\ 
        $\tau_2$ &  $(5, 5, 16)$ & $40$  \\ 
        \hline
    \end{tabular} 
    \caption{A set of segmented self-suspending tasks for demonstrating the misconception in the literature when analyzing the schedulability of task $\tau_k$ under 
segmented fixed-priority scheduling with periodic enforcement in Section~\ref{sec:wrong-periodic}.}
    \label{table:ex-periodic}
\end{table}


\begin{figure}[t]
\centering
\def\uxfpga{0.3cm}
\scalebox{0.91}{
\begin{tikzpicture}[x=\uxfpga,y=\uy,auto, thick]
    \draw[->] (0,0) -- coordinate (xaxis) (42,0) node[anchor=north]{$t$};
    \node[anchor=east] at (0, 0.5) {$\tau_2$};
    \node[anchor=east] at (0, 2.5) {$\tau_1$};

    \foreach \x in {0,5,...,40}{
		\draw[-,below](\x,0) -- (\x,-0.3) node[] {\pgfmathtruncatemacro\yi{\x} \yi};
    }
    \foreach \x in {0,1,...,40}{
		\draw[-,very thin,lightgray, dashed](\x,0.3) -- (\x,4);
    }     

	\begin{scope}[shift={(0,0)}]        
        \draw[] (0,2) -- (40,2);
        \draw[->] (0,0) -- (0,1.75);
        \draw[->] (10,0) -- (10,1.75);
        \foreach \y in {0.3,0.5,0.7}{
			\draw[] (5,\y) -- (10,\y);
		}
        \draw[<-,thin] (0,1.3) -- (3.3,1.3);
        \draw[->,thin] (6.6,1.3) -- (10,1.3);
        \node[anchor=east] at (6.6, 1.49) {offset};
        \draw[<-,red] (40,0) -- (40,1.2);
        \node[anchor=east,red] at (41.3, 1.49) {miss};
        \node[task7, minimum width=5*\uxfpga, anchor=south west] at (0, 0){\footnotesize $C_2^1$};         
        \node[task7, minimum width=15*\uxfpga, anchor=south west] at (15, 0){\footnotesize $C_2^2$};
	\end{scope}
	
	\begin{scope}[shift={(0,2)}]
        \draw[->] (0,0) -- (0,1.75);
        \draw[->] (30,0) -- (30,1.75);
        \node[task7, minimum width=10*\uxfpga, anchor=south west] at (5, 0){\footnotesize};         
        \node[task7, minimum width=10*\uxfpga, anchor=south west] at (30, 0){\footnotesize};
	\end{scope}
  \end{tikzpicture}}       
  \caption{A schedule to release the two tasks in Table~\ref{table:ex-periodic} simultaneously. Task $\tau_2$ in this schedule has longer worst-case response time than the incorrect schedulability analysis used in   \cite{RTSS-KimANR13,DBLP:journals/ieicet/DingTT09}.
}
  \label{fig:counterexample-FP-segment-level}
\end{figure}

%\begin{figure}[t]
%\begin{center}
%   \begin{tikzpicture}[y=\uy, font=\sffamily,thick]
     
       
       
        \begin{scope}[shift={(0,0)}]
       \draw[->] (0,0)node[anchor=east,align=center] {$\tau_2$} -- coordinate (xaxis) (8.5,0);
      	\foreach \x in {3}{
      
	 	\node[task7, minimum width=6*\uy,
			anchor=south west] at ( \x, 0){\footnotesize $C_2^2$};
	}
	\foreach \x in {0}{

      		 \node[task7, minimum width=2*\uy,
anchor=south west] at ( \x, 0){\footnotesize $C_2^1$};
	 	
	}
	
	\foreach \x in {0}{
		\draw[->](\x,0) -- (\x,2)
	 		node[above] {};
	}
	\foreach \x in {2}{
		\draw[->](\x,0) -- (\x,2)
	 		node[above] {};
	}
	
	\foreach \x in {8}{
		\draw[->,red](\x,1.5) node[anchor=south] {\textit{miss}}  -- (\x,0)
			node[] {$\times$};
	 		
	}
	\foreach \x in {0,1,...,8}{
		\draw[-,below](\x,0) -- (\x,-0.1)
node[] {\pgfmathtruncatemacro\yi{5*\x} \yi};

			
	 		
	}
       \end{scope}
    
      \begin{scope}[shift={(0,2.4)}]
   
      	\draw[->](0,0) -- (0,1.5);
%\draw[->](2,0) -- (2,1);
\draw[->](6,0) -- (6,1.5);

	\foreach \x in {1,6}{
       		\node[task7, minimum width=4*\uy,
			anchor=south west] at ( \x, 0){$C_1$};
	 	
	}
	\draw[->] (0,0)node[anchor=east] {$\tau_1$} -- coordinate (xaxis) (8.5,0);


	
	
       \end{scope}
 %\draw[dotted] (10,0) -- (10,3.5);
    %  \draw[dotted] (15,0) -- (15,3.5);
     % \draw [<->] (10,3.5) -- (15,3.5)
     % 	node[anchor=south,pos=0.5] {$carry$-$in$};


      \end{tikzpicture}
%\end{center}
%\caption{A schedule to release the two tasks in Section~\ref{sec:wrong-periodic} simultaneously at time $0$.}
%\label{fig:counterexample-FP-segment-level}
%\end{figure}

{\bf Consequences:} The priority assignment algorithms in \cite{RTSS-KimANR13,DBLP:journals/ieicet/DingTT09} use the above unsafe schedulability test to verify the priority assignments. Therefore, their results are flawed due to the unsafe schedulability test.

{\bf Solutions:} This requires us to revisit the schedulability test of a given segmented fixed-priority assignment. As discussed in Section~\ref{sec:static-period-enforce}, this can be observed as a reduction to 
the generalized multiframe (GMF) task model introduced by Baruah et al.~\cite{baruah1999generalized}. However, most of the existing fixed-priority scheduling results for the GMF task model assume a unique priority level \emph{per task}. To the best of our knowledge, the only results that can be applied for a unique level \emph{per computation segment} are the utilization-based analysis in \cite{DBLP:journals/corr/ChenHL15b,huang2015mode}. 

% \mysection{Incorrect Scheduling with Slack Enforcement}
% \label{sec:wrong-slack}


\mysection{Incorrect Combination by Using Release Jitter without Converting Suspension into Computation}
\label{sec:wrong-jitter-convert-sporadic}


We now explain a misconception that treats the higher-priority
self-suspension tasks by introducing safe release jitters and analyzes
the response time of task $\tau_k$ by accounting the self-suspending behavior explicitly.  Consider
the example listed in Table~\ref{table:ex-wrong-jitter-split}.  Task
$\tau_1$ obviously meets its deadline.  Task $\tau_2$ can be validated
to meet its deadline by using the split approach, \ie, $8 + 12+ 8 =
28$. The jitter of task $\tau_2$ is hence at most $28-2\times 3 = 22$.

Since the jitter of task $\tau_2$ is small, \ie,
$\ceiling{\frac{t+22}{T_3}} = 1$ for any $0 \leq t \leq 39$, we can
conclude that there is only one active job of task $\tau_2$ in time
interval $(a, a+39]$, in which a job of task $\tau_3$ arrives at time
$a$. Theorem 2 in \cite{ecrts15nelissen} exploited the above property
and converted task $\tau_2$ to an ordinary sporadic task, denoted as
task $\tau_2'$ here, with jitter equal to $22$. By the above
discussion, in our setting in Table~\ref{table:ex-wrong-jitter-split},
there is only one job of task $\tau_2'$ that can interfere with a job
of task $\tau_3$.

By the above conversion, the interfering job of task $\tau_2'$
hits either the first or the second computation segment of task
$\tau_3$. In both cases, that computation segment of task $\tau_3$ can be
finished within $19$ time units, \ie, $3+6 +
\ceiling{\frac{19}{10}}\times 5 = 19$. For the other segment of task
$\tau_3$ that is not interfered by the job of task $\tau_2'$, the
segment can be finished within $3+5=8$ time units.  Therefore, the
above analysis concludes that the worst-case response time of task
$\tau_3$ is $19+S_3^1 + 8 = 31$.

However, the perfectly legal schedule in
Figure~\ref{fig:counterexample-wrong-jitter-split} disproves this.  In
that schedule, the response time of task $\tau_3$ is $36$.

\begin{table} [t]
\centering
    \begin{tabular}{|c|c|c|c|}
 \hline
        & $(C_i^1, S_i^1, C_i^2)$ &  $D_i$ & $T_i$ \\ 
        \hline
        $\tau_1$ & $(5, 0, 0)$ &  $10$ & $10$\\ 
        $\tau_2$ & $(3, 12, 3)$ &  $28$ & $1000$\\ 
        $\tau_3$ & $(3, 4, 3)$ & $35$ & $1000$  \\ 
        \hline
    \end{tabular} 
    \caption{A set of segmented self-suspending tasks for demonstrating the misconception when analyzing the schedulability of task $\tau_k$ by combining the release jitter approach (for the higher-priority interferring tasks) and  explicit self-suspension behavior (for the interfered task $\tau_k)$ in Section~\ref{sec:wrong-jitter-convert-sporadic}.}
    \label{table:ex-wrong-jitter-split}
\end{table}


\begin{figure}[t]
\centering
\def\uxfpga{0.3cm}
\scalebox{0.91}{
\begin{tikzpicture}[x=\uxfpga,y=\uy,auto, thick]
    \draw[->] (0,0) -- coordinate (xaxis) (42,0) node[anchor=north]{$t$};
    \node[anchor=east] at (0, 0.5) {$\tau_3$};
    \node[anchor=east] at (0, 2.5) {$\tau_2$};
    \node[anchor=east] at (0, 4.5) {$\tau_1$};

    \foreach \x in {0,5,...,40}{
		\draw[-,below](\x,0) -- (\x,-0.3) node[] {\pgfmathtruncatemacro\yi{\x} \yi};
    }
    \foreach \x in {0,1,...,40}{
		\draw[-,very thin,lightgray, dashed](\x,0.3) -- (\x,6);
    }     

      \draw[->] (0,0) -- (0,1.75);
        \foreach \y in {0.3,0.5,0.7}{
			\draw[] (16,\y) -- (20,\y);
		}
        \draw[<-,red] (35,0) -- (35,1.2);
        \node[anchor=east,red] at (35.3, 1.49) {miss};
        \node[task7, minimum width=2*\uxfpga, anchor=south west] at (8, 0){};         
        \node[task7, minimum width=1*\uxfpga, anchor=south west] at (15, 0){};         
        \node[task7, minimum width=2*\uxfpga, anchor=south west] at (28, 0){};         
        \node[task7, minimum width=1*\uxfpga, anchor=south west] at (35, 0){};         
	\begin{scope}[shift={(0,2)}]        
        \draw[] (0,0) -- (40,0);
        \draw[->] (0,0) -- (0,1.75);
        \draw[<-] (28,0) -- (28,1.75);
        \foreach \y in {0.3,0.5,0.7}{
			\draw[] (8,\y) -- (20,\y);
		}
        \node[task7, minimum width=3*\uxfpga, anchor=south west] at (5, 0){\footnotesize $C_2^1$};         
        \node[task7, minimum width=3*\uxfpga, anchor=south west] at (25, 0){\footnotesize $C_2^2$};         
	\end{scope}
	
	\begin{scope}[shift={(0,4)}]
        \draw[] (0,0) -- (40,0);
        \draw[->] (0,0) -- (0,1.75);
        \draw[<->] (10,0) -- (10,1.75);
        \draw[<->] (20,0) -- (20,1.75);
        \draw[<->] (30,0) -- (30,1.75);
        \draw[<->] (40,0) -- (40,1.75);
        \node[task7, minimum width=5*\uxfpga, anchor=south west] at (0, 0){\footnotesize};         
        \node[task7, minimum width=5*\uxfpga, anchor=south west] at (10, 0){\footnotesize};
        \node[task7, minimum width=5*\uxfpga, anchor=south west] at (20, 0){\footnotesize};
        \node[task7, minimum width=5*\uxfpga, anchor=south west] at (30, 0){\footnotesize};
	\end{scope}
  \end{tikzpicture}}       
\caption{A schedule to release the three tasks in
  Table~\ref{table:ex-wrong-jitter-split} simultaneously. This schedule shows that the self-suspension behavior of task $\tau_2$ matters, as explained in Section~\ref{sec:wrong-jitter-convert-sporadic}.}
  \label{fig:counterexample-wrong-jitter-split}
\end{figure}


{\bf Consequences:} This example shows that the analysis in Section VI by Nelissen et al. \cite{ecrts15nelissen} is flawed.  

{\bf Solutions:} Each computation segment of a higher-priority task
should be treated as an individual sporadic task with jitter. That is,
the treatment in Section~VI by Nelissen et al. \cite{ecrts15nelissen} remains valid
if each computation segment of a higher-priority task $\tau_i$ is
converted to a sporadic task with proper jitter.


%%% Local Variables:
%%% mode: latex
%%% TeX-master: "JRTS/JRTS.tex"
%%% End:
