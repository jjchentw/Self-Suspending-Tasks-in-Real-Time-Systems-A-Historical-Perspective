
\section{General Design and Analysis Strategies}
\label{sec:review}

Self-suspending tasks have been widely studied in the literature and several solutions have been proposed over the years for 
analyzing their schedulability and building efficient schedules. In this section, we provide an overview of the different 
strategies commonly adopted in the state-of-the-art approaches to analyze and solve the self-suspending task scheduling problem. Although such strategies are
correct in essence, many published results based on those generic analysis frameworks have been corrupted by a set of 
misconceptions which led to incorrect solutions. In an attempt to stop the propagation of erroneous results, a precise 
description of the misunderstandings about the self-suspending task model, together with the demonstration of 
counter-intuitive results is provided in Section~\ref{sec:misconceptions}.


Let $\tau_k$ be a self-suspending task.
As illustrated in the example below, the worst-case response time, and hence the schedulability, of task $\tau_k$ is impacted not only by 
the self-suspension behavior of $\tau_k$ but also by the self-suspension behavior of  the higher-priority jobs.

%\begin{example}
%\label{ex:taui_ss}
%Let $\tau_2$ be a self-suspending task characterized as follows $((C,S,C),T,D)$ and $\tau_1$ be a non-self-suspending task defined by $(C,T,D)$. Assume that a fixed priority scheduling algorithm is used to schedule those tasks and that $\tau_1$ has a higher priority than $\tau_2$. If one should ignore the self-suspension behavior of $\tau_2$ by assuming an equivalent non-self-suspending task $\tau_2' = (C+C,T,D )$, the worst-case response time of $\tau_2'$ computed with the usual RTA \cite{...} would be .... Yet, as illustrated on Figure~\ref{fig:ex_sched:taui_ss}, the worst-case response time of $\tau_2$ is ... and is obtained when (i) $\tau_2$ experiences its maximum self-suspending time and (ii) $\tau_1$ is synchronously released with the second computation segment of $\tau_2$. 
%\end{example}

\begin{example}
\label{ex:tauh_ss} 
Let $\tau_2$ be an ordinary sporadic task with the following attributes $(C_2, T_2, D_2)=(5,10,10)$ and $\tau_1$ be a self-suspending  sporadic
task defined as $((C_1^1, S_1^1, C_1^2), T_1, D_1)=((1,2,1),8,8)$. Assume that $\tau_2$ and $\tau_1$ are scheduled with a fixed-priority scheduling algorithm on a uniprocessor system
and that $\tau_1$ has higher priority than $\tau_2$. If one should ignore the self-suspending behavior of $\tau_1$ by 
assuming a self-suspending task $\tau_1' = (C_1', T_1', D_1')=(2, 8, 8)$, the worst-case response time of $\tau_2$ computed 
with the standard response time analysis (RTA) in Eq.~(\ref{eq:rta}) would be $R_2 = 7$. However, as illustrated in Figure~\ref{fig:simple-example-suspension},
due to the self-suspension of $\tau_1$, the actual worst-case response time of 
$\tau_2$ is $8$ and is obtained when (i) $\tau_1$ experiences its maximum self-suspending time and (ii) $\tau_2$ is 
synchronously released with the second computation segment of $\tau_1$.
\hfill\myendproof  
\end{example}


\begin{figure}[t]
\centering
\def\uxfpga{0.4cm} 
\scalebox{0.8}{
	\begin{tikzpicture}[x=\uxfpga,y=\uy,auto, thick]
    \draw[->] (0,0) -- coordinate (xaxis) (15,0) node[anchor=north]{$t$};
    \draw[] (0,2) -- coordinate (xaxis) (12,2) node[anchor=north]{};
    \foreach \x in {0,1,...,12}{
      \draw[-,below](\x,0) -- (\x,-0.3) node[] {\pgfmathtruncatemacro\yi{\x} \yi};
    }
    \foreach \x in {0,3,4,8,9,10}{
      \draw[-,very thin,lightgray, dashed](\x,0.3) -- (\x,4);
    }	
    
    \begin{scope}[shift={(0,0)}]
      \node[anchor=east] at (0, 0.5) {$\tau_2$};
      \draw[->] (3,0) -- (3,1.75);
      \draw[<-,thin,red] (3,1.3) -- (7.5,1.3);
      \draw[->,thin,red] (8.5,1.3) -- (11,1.3);


        \node[task7, minimum width=4*\uxfpga, anchor=south west] at (4, 0){};
        \node[task7, minimum width=\uxfpga, anchor=south west] at (10, 0){};

      \draw[] (11.05,0) -- (11.05,1.5);
      \node[anchor=east,red] at (8.5, 1.39) {$8$};
      \draw[dotted] (11.5,0.5) -- (12.3,0.5); 
    \end{scope}
    
    \begin{scope}[shift={(0,2)}]
      \node[anchor=east] at (0, 0.5) {$\tau_1$};
      \foreach \x in {0,8}{ 
        \draw[->] (\x,0) -- (\x,1.75);        		
      }
      \foreach \x in {0,3,8,9}{ 
        \node[task7, minimum width=\uxfpga, anchor=south west] at (\x, 0){};
      }
    \end{scope}                
  \end{tikzpicture}}
\caption{Self-suspension can cause substantial schedulability degradation for Example \ref{ex:tauh_ss}.}
\label{fig:simple-example-suspension}
\end{figure}


As discussed in details in Section~\ref{sec:hardness}, performing the timing analysis for a set of self-suspending tasks has been proven extremely complex and even intractable in the general case. For those reasons, most works adopt some common strategies to simplify the worst-case response time analysis of self-suspending tasks. We present those strategies in Section~\ref{sec:model-interferred} and Section~\ref{sec:model-interfering} by decoupling the modeling of the task under analysis (\ie, $\tau_2$ in the examples above) and the task interfering with the analyzed task, respectively. Section~\ref{sec:release-enforce} presents  release enforcement mechanisms to reduce the impact due to self-suspension. Section~\ref{sec:multiprocessor-HRT} will shortly discuss how to handle self-suspending tasks in multiprocessor systems.


Note that we implicitly assume uniprocessor systems in these sections. The multiprocessor case is covered in Section~\ref{sec:multiprocessor-HRT}. In Sections~\ref{sec:model-interferred},~\ref{sec:model-interfering},~and~\ref{sec:release-enforce}, in most cases, we will use fixed-priority scheduling to explain the strategies. Therefore, we implicitly consider the timing analysis for task $\tau_k$, in which $hp(k)$ is the set of higher-priority tasks, if fixed-priority scheduling is considered.

\subsection{Modeling the Interfered Task}
\label{sec:model-interferred}

Two main strategies have been proposed in the literature to simplify the modeling of a self-suspending task $\tau_k$ during its worst-case response time analysis:
\begin{itemize}
\item the self-suspension \emph{oblivious} approach, which models the suspension intervals of $\tau_k$ as if they were usual execution time (Section~\ref{sec:model-interferred-oblivious});
\item the \emph{split} approach, which computes the worst-case response time of each computation segment of $\tau_k$ as if they were independent tasks (Section~\ref{sec:model-interferred:split}).
\end{itemize}
Strategies combining both approaches have also been investigated as discussed in Section~\ref{sec:model-interferred-hybrid}.
At the date of the writing of this document and to the best of the authors' knowledge, no tractable solution has been found to compute the exact worst-case interference suffered by a segmented self-suspending task. 




\subsubsection{Modeling suspension as computation}
\label{sec:model-interferred-oblivious}

This strategy is sometimes called ``joint'' \cite{bletsas:thesis} but often referred to as the \emph{suspension-oblivious} approach in the literature. It consists in assuming that the self-suspending task $\tau_k$ continues executing on the processor when it self-suspends. Its suspension intervals are thus considered as being preemptible. From an analysis perspective, it is equivalent to replacing the self-suspending task $\tau_k$ by a non-self-suspending task $\tau_k' = (C_k + S_k, D_k, T_k)$ with a worst-case execution time equal to $C_k + S_k$. 

%Since the suspension intervals are assumed to be preemptible
Converting the suspension time of task $\tau_k$ into computation time can become very pessimistic for \emph{segmented} self-suspending tasks. This is especially true when (i) its total self-suspending time $S_k$ is much larger than its worst-case execution time $C_k$ and/or (ii) the lengths of $\tau_k$'s suspension intervals are larger than the periods of (some of) the interfering tasks. 

\begin{example}
\label{ex:suspension-as-comput}   
Using the task set presented in Table~\ref{table:static-example} as an example, task $\tau_3$ would be transformed in a non-self-suspending task $\tau_3' = (7, 15, 15)$. Task $\tau_3'$ is obviously not schedulable after this transformation since the total utilization of $\tau_1$, $\tau_2$ and $\tau_3'$ is given by $\frac{2}{5} + \frac{2}{10} + \frac{7}{15} = \frac{16}{15} > 1$. Yet, the self-suspending task $\tau_3$ is schedulable as it will be shown in Section~\ref{sec:model-interferred:split}.
\hfill\myendproof  
\end{example}

Nevertheless, although non-intuitive, this modeling strategy is an \emph{exact} solution to compute the WCRT of \emph{dynamic} self-suspending tasks. If the computation segments and suspension intervals of $\tau_k$ interleave such that $\tau_k$ self-suspends only between the arrival of higher priority jobs
(\ie,  a computation segment of $\tau_k$ is started whenever a higher priority job is released), then the resulting schedule would be similar if $\tau_k$ was indeed executing on the processor during its self-suspensions. This is used in all the analyses for dynamic self-suspension task models, \eg, \cite{LiuChen:rtss2014,huangpass:dac2015,MingLiRTCSA1994,RTCSA-KimCPKH95,RTAS-AudsleyB04,ECRTS-AudsleyB04}.
Therefore, when there is no knowledge about how many times, when and for how long $\tau_k$ may self-suspend in each 
self-suspending interval, modeling the self-suspending time of $\tau_k$ as execution time provides the exact worst-case response 
time for $\tau_k$.


%This is the simplest and the most pessimistic strategy. It converts all self-suspending time into computation time. Such a strategy is also referred to as \emph{suspension-oblivious} analysis in the literature. That is, we can consider the execution time of task $\tau_i$ to be $C_i+S_i$. After this conversion, we only have ordinary sporadic real-time tasks. Therefore, all the existing results for sporadic task systems can be applied. %The proof can be done with the following simple interpretation: The suspension of a job may make the processor idle. If two jobs suspend at the same time and the processor idles in a certain time interval in the actual schedule, it can be imagined that one of these two jobs has shorter execution time (than its worst-case execution time $C_i+S_i$). Such earlier completion does not affect the schedulability analysis. Therefore, putting $C_i+S_i$ as the worst-case execution time for every task $\tau_i$ is a very safe analysis for both dynamic- and static-scheduling policies.  Such an approach has been widely used as the baseline of more accurate analyses in the literature.

%With this schedulability test, it is easy to see that none of the example task sets can be classified as feasible since $\frac{1}{2} + \frac{5+5}{20} + \frac{1}{D_\gamma} > 1$ and $\frac{2}{5}+\frac{2}{10}+\frac{1+5+1}{15} > 1$.

\subsubsection{Modeling each computation segment as an independent task}
\label{sec:model-interferred:split}

An alternative is to individually compute the WCRT of each of the computation segments of task $\tau_k$ \cite{bletsas:thesis,PH:rtss98}. 
The WCRT of $\tau_k$ is then upper-bounded by the sum of the segments' worst-case response times added to $S_k$, the maximum length of the overall
self-suspending intervals. 
 
Let $R_k^j$ denote the worst-case response time of the computation segment $C_k^j$. The schedulability test of task $\tau_k$ consists 
in verifying whether $\sum_{j=1}^{m_k} R_k^j + \sum_{j=1}^{m_k-1} S_k^j \leq D_k $. 

\begin{table}[t]
\centering
    \begin{tabular}{|c|c|c|c|}
 \hline
        & $(C_i^1, S_i^2, C_i^2)$ &  $D_i$ & $T_i$\\ 
        \hline
        $\tau_1$ & (2, 0, 0) &  5 & 5\\ 
        $\tau_2$ &  (2, 0, 0) & 10 & 10 \\ 
        $\tau_3$ & (1, 5, 1) & 15  & 15\\
%        $\tau_4$ & (3, 0, 0) & ? & $\infty$\\
        \hline
    \end{tabular} 
    \caption{Example of a segmented self-suspending task set, used in Examples~\ref{ex:suspension-as-comput} and \ref{ex:suspension-as-split-1}.}
    \label{table:static-example}
\end{table}

\begin{example}
\label{ex:suspension-as-split-1}   
Let us use the task set presented in Table~\ref{table:static-example} as an example. The worst-case response times of $C_3^1=1$ and $C_3^2=1$ are both $5$ by using the usual RTA in Eq.~(\ref{eq:rta}) for ordinary sporadic real-time tasks. Therefore, we know that the worst-case response time of task $\tau_3$ is at most $R_3^1 + R_3^2 + S_3 = 5 + 5 + 5 = 15$.
\hfill\myendproof  
\end{example}

The above test can be fairly pessimistic, especially when $S_k$ is short. 
\begin{example}
\label{ex:suspension-as-split-2}   
Imagine that $S_3$ is decreased from $5$ to $1$ in the previous example. This analysis still considers that both computation segments suffer from the worst-case interference from the two higher-priority tasks. It then returns $R_3^1 + R_3^2 + S_3 = 5 + 5 + 1 =11$ as the (upper bound on the) worst-case response time of $\tau_3$. Yet, the suspension oblivious approach mentioned in Section~\ref{sec:model-interferred-oblivious}, tells us that the worst-case response time of $\tau_3$ is at most $9$.
\hfill\myendproof  
\end{example} 

\subsubsection{Hybrid approaches}
\label{sec:model-interferred-hybrid}

Both methods discussed above in Sections~\ref{sec:model-interferred-oblivious} and~\ref{sec:model-carry-in} have their pros and cons. The \emph{joint} or \emph{suspension-oblivious} in Section~\ref{sec:model-interferred-oblivious} approach has the advantage of
respecting the minimum inter-arrival times (or periods) of the higher priority tasks during the schedulability 
analysis of $\tau_k$. However, it has the disadvantage of assuming that the suspension intervals are preemptible by converting them as computation. 
This renders the analysis pessimistic as it accounts for non-existing interference. The \emph{split} approach does not assume  
preemptible suspension intervals but  considers a worst-case response time for each computation segment independently. Yet, the respective
release patterns of interfering tasks leading to the worst-case response time of each computation segment may not not be compatible with each other.

As shown with the above examples, the joint and split approaches are not comparable in the sense that none 
of them always outperforms the other. Yet, since both provide an upper bound on the worst-case response time of $\tau_k$, one can 
simply take the minimum response time value obtained with any of them. However, as proposed in \cite{bletsas:thesis} (Chapter 5.4), 
it is also possible to combine their respective advantages and hence reduce the overall pessimism of the analysis. 
The technique proposed in \cite{bletsas:thesis}, for tasks of the \emph{segmented} model,
consists in dividing the self-suspending task $\tau_k$ (that is under analysis) into several blocks of consecutive 
computation segments. The suspension intervals between computation segments pertaining to the same block are modeled as execution time 
like in the ``joint" approach. The suspension intervals situated between blocks are ``split". The worst-case response time is then computed for each 
block independently and $\tau_k$'s WCRT is upper bounded by the sum of the block's WCRTs added to the length of the split suspension 
intervals. This provides a tighter bound on the WCRT, especially if one considers all possible block sequence decompositions of $\tau_k$. 
It is clearly a process with exponential time complexity.
% It was however shown to be applicable to task systems of realistic sizes.

\subsubsection{Exact schedulability analysis}

As already mentioned in Section~\ref{sec:model-interferred-oblivious}, the self-suspension oblivious approach is an exact analysis for dynamic 
self-suspending tasks assuming that there is only one self-suspending task $\tau_k$ and all the interfering tasks do not self-suspend. 
There is no work providing an exact schedulability analysis for any other cases under the dynamic self-suspending task model.

The problem of the schedulability analysis of segmented self-suspending tasks has been treated in~\cite{LR:rtas10,ecrts15nelissen}, 
again assuming only one self-suspending task $\tau_k$. The proposed solutions are based on the notion of the critical instant. 
That is, they aim to find the instant at which, considering the state of the system, an execution request for $\tau_k$ will 
generate the largest response time. Unfortunately, the analysis in \cite{LR:rtas10} has been proven to be flawed in~\cite{ecrts15nelissen}. 
Further details on the reasons are provided in Section~\ref{sec:wrong-critical}.


\subsection{Modeling the Interfering Tasks}
\label{sec:model-interfering}



\subsubsection{Suspension oblivious analysis}
\label{sec:model-interfering-oblivious}

Similarly to the task under analysis, the simplest modeling strategy for the interfering tasks is the suspension oblivious approach, which consists in converting all the suspension times of those tasks into computation times. Each task $\tau_i$ is thus modeled by a non-self-suspending task $\tau_i' = (C'_i, D_i, T_i)$ with a WCET $C'_i = C_i+S_i$. After that conversion, the interfering tasks therefore become a set of usual non-self-suspending sporadic real-time tasks. Although the simplest, it is also the most pessimistic approach. It indeed considers that the suspension intervals of each interfering task $\tau_i$ are causing interference on the task $\tau_k$ under analysis. Yet, suspension intervals truly model durations during which $\tau_i$ stops executing on the processor and hence cannot prevent the execution of $\tau_k$ or any other lower priority job.


\subsubsection{Modeling self-suspensions with carry-in jobs}
\label{sec:model-carry-in}

If all the higher-priority jobs/tasks are ordinary sporadic jobs/tasks without any self-suspensions, then
the maximum number of interfering jobs that can be released by an interfering (ordinary) sporadic task $\tau_i$ in a window of length $t$, is upper bounded by $\left\lceil \frac{t}{T_i} \right\rceil$ in fixed-priority scheduling and by $\left\lfloor \frac{t}{T_i} \right\rfloor$ in EDF scheduling. The interfering workload is then given by $\sum_{\forall \tau_i \in hp(k)} \left\lceil \frac{t}{T_i} \right\rceil C_i$ for fixed priority scheduling and by $\sum_{\forall \tau_i \in \tau \setminus \tau_k} \left\lfloor \frac{t}{T_i} \right\rfloor C_i$ for EDF scheduling. This assumes that each interfering job asks for the processor as soon as it is released, thereby preventing the task $\tau_k$ under analysis to execute.

With self-suspending tasks however, the \emph{computation segment} of an interfering job may not require a direct access to the processor as it can be delayed by its suspension intervals. Hence, a job of task $\tau_i$ released before the release of a job of task $\tau_k$ may have all its execution time $C_i$ delayed by its suspension intervals so as to entirely interfere with $\tau_k$. 
This is clearly visible on the example schedule of Figure~\ref{fig:counterexample-segmented}.
Such a job of $\tau_i$, released before the job of $\tau_k$ under analysis, but interfering with the execution of $\tau_k$, is called a \emph{carry-in job}.

In the worst case, each interfering task $\tau_i$ releases one carry-in job (assuming that they all respect their deadlines and that $D_i \leq T_i$). This extra-workload, which can be up to $C_i$, has been integrated in the schedulability test for self-suspending tasks in \cite{huangpass:dac2015,LiuChen:rtss2014} by greedily adding one interfering job to the interfering workload released by each task $\tau_i$.


%In the \emph{dynamic self-suspending task model}, the above analysis implies that the higher-priority jobs arrived after time $t_k$ \emph{should not} suspend themselves to create the maximum interference. Therefore, suppose that the first arrival time of task $\tau_i$ after $t_k$ is $t_i$, \ie, $t_i \geq t_k$. Then, the demand of task $\tau_i$ released at time $t \geq t_i$ is $\left\lceil \frac{t-t_i}{T_i} \right\rceil C_i$. So, we just have to account for the demand of the carry-in job of task $\tau_i$ executed between $t_k$ and $t_i$. The workload of the carry-in job can be up to $C_i$ (due to the assumption $D_i \leq T_i$), but can also be characterized in a more precise manner. The approaches in this category are presented in \cite{huangpass:dac2015,LiuChen:rtss2014} by greedily counting $C_i$ in the carry-in job. 


\subsubsection{Modeling self-suspensions as release jitter}
\label{sec:model-interfering-jitter}

Another, more accurate, way to model the phenomena described above to use the concept of \emph{release jitter}. It basically considers 
that the computation segments of each task $\tau_i$ are not released in a purely periodic manner but are instead subject to release jitter. 
Hence the first interfering job of $\tau_i$ may have its computation segment pushed as far as possible from the actual release of the job 
due to its suspension behavior, while all the jobs released afterward may directly start with their computation segments and never 
self-suspend (see $\tau_2$ in Figure~\ref{fig:counterexample-dynamic} for an example). Let $J_i$ denote that jitter on $\tau_i$'s computation segment release. 
It was proven in \cite{ecrts15nelissen,BletsasReport2015} that $J_i$ is upper-bounded by $R_i-C_i$ where $R_i$ is the WCRT of $\tau_i$. 
If an optimal priority assignment must be computed for a fixed-priority task set using the Audsley's optimal priority assignment 
algorithm~\cite{Audsley1991aOPA}, one can pessimistically assume that $J_i$ is equal to $D_i-C_i$ \cite{huangpass:dac2015,Raj:suspension1991} 
as long as all the interfering tasks, \ie $\forall \tau_i \in hp(k)$ in fixed-priority scheduling, are schedulable, \ie, $R_i \leq D_i$.


\begin{table}[t]
\centering
    \begin{tabular}{|c|c|c|c|c|}
 \hline
        & $C_i$ &  $S_i$&  $D_i$ & $T_i$\\ 
        \hline
        $\tau_1$ & 1 & 0 &  2 & 2\\ 
        $\tau_2$ &  5&  5& 20 & 20 \\ 
        $\tau_3$ & 1 & 0  & 50 & $\infty$ \\ 
        \hline
    \end{tabular} 
    \caption{Example of a dynamic self-suspending task set used in Examples~\ref{ex:suspension-jitter} and~\ref{ex:suspension-blocking}.}
    \label{table:dynamic-example}
\end{table}

For a fixed-priority task set under the dynamic self-suspension model, the WCRT of $\tau_k$ is upper bounded by the smallest value $R_k$ 
larger than $0$ such that 
\begin{equation*}
R_k=C_k+ \sum_{\forall \tau_i \in hp(k)} \left\lceil \frac{t + J_i}{T_i} \right\rceil C_i
\end{equation*} 

The response time analysis for EDF can be similarly adapted.

\begin{example}
\label{ex:suspension-jitter}    
Consider the fixed priority task set presented in Table~\ref{table:dynamic-example}. In this case, $\tau_1$ is the highest priority task and does not self-suspend. Therefore, its WCRT is $R_1 = C_1$ and $J_1 = R_1 - C_1 = 0$. The jitter $J_2$, however, is upper bounded by $D_2 - C_2 =15$. The WCRT of task $\tau_3$ is thus upper bounded by the minimum $t$ larger than $0$ such that 
$$t=C_3+ \sum_{i=1}^2\left\lceil \frac{t + J_i}{T_i} \right\rceil C_i = 1+\left\lceil \frac{t}{2} \right\rceil 1 +\left\lceil \frac{t+15}{20} \right\rceil 5.$$ 
The above equality holds when $t=22$. Therefore, the WCRT of task $\tau_{3}$ is upper bounded by $22$.
\hfill\myendproof  
\end{example}

Note that several solutions proposed in the literature \cite{ECRTS-AudsleyB04,RTAS-AudsleyB04,RTCSA-KimCPKH95} for modeling the 
self-suspending behavior of the intefering tasks as release jitter, are flawed. Those analyses usually assume that $J_i$ can be 
upper-bounded by the total self-suspension time $S_i$ of $\tau_i$. This is usually wrong. A detailed discussion on this matter is 
provided in Section~\ref{sec:wrong-jitter-dynamic}. 

\subsubsection{Modeling self-suspensions as blocking}
\label{sec:model-interfering-blocking}

In her book~\cite[Pages 164-165]{Liu:2000:RS:518501}, Jane W.S. Liu proposed an approach to quantify the interference
higher-priority tasks by setting up the ``blocking time" induced by the self-suspensions of the interfering tasks on the 
task $\tau_k$ under analysis. This solution, limited to fixed-priority scheduling policies, considers that a job of 
task $\tau_k$ can suffer an extra delay on its completion due to the self-suspending behavior of each task involved in its 
response time. This delay, denoted by $B_k$, is upper bounded by 
\begin{equation*}
B_k=S_k+\sum_{\forall \tau_i \in hp(k)} b_i
\end{equation*}
where (i) $S_k$ accounts for the contribution of the suspension intervals of the task $\tau_k$ under analysis in a similar manner to 
what has already been discussed in Section~\ref{sec:model-interferred-oblivious}, and (ii) $b_i=min(C_i, S_i)$ accounts for the contribution of each higher priority task $\tau_i$ in $hp(k)$. This equivalent ``blocking time" $B_k$ can then be used to perform a utilization-based schedulability test. For instance, using the linear utilization test by Liu and Layland \cite{Liu_1973} and assuming that the tasks are indexed in a decreasing priority order, ensuring that the condition
\begin{equation*}
\frac{C_k+B_k}{T_k} + \sum_{\forall \tau_i \in hp(k)} U_i \leq k (2^{\frac{1}{k}}-1)
\end{equation*}
is respected for all tasks, is a sufficient schedulability test for fixed-priority task sets.

This blocking time can also be integrated in the WCRT analysis for fixed priorities. The WCRT of $\tau_k$ is then given by the smallest value of $R_k$ larger than $0$ such that
\begin{equation*}
R_k = B_k + C_k + \sum_{\forall \tau_i \in hp(k)} \left\lceil \frac{R_k}{T_i} \right\rceil C_i
\end{equation*}


Note that even though \cite{Liu:2000:RS:518501} discusses the intuition behind this modeling strategy, it does not provide any actual proof of its correctness. 
However, the correctness of that approach has since been proven in \cite{ChenHuangNelissen}. 

\begin{example}
\label{ex:suspension-blocking}    
Consider the task set presented in Table~\ref{table:dynamic-example} to illustrate the above analysis. In this case, $b_1 = 0$ and $b_2 = 5$. Therefore, $B_3 = 5$. So, the worst-case response time of task $\tau_3$ is upper bounded by the minimum $t$ larger than $0$ such that 
\begin{align*}
t & = B_3+C_3+\sum_{i=1}^2 \left\lceil \frac{t}{T_i} \right\rceil C_i = 6+\left\lceil \frac{t}{2} \right\rceil 1 +\left\lceil \frac{t}{20} \right\rceil 5.
\end{align*} 
This equality holds when $t=32$. Therefore, the WCRT of task $\tau_{3}$ is upper bounded by $32$.
\hfill\myendproof  
\end{example}



Devi (in Theorem 8 in \cite[Section 4.5]{DBLP:conf/ecrts/Devi03}) extended the above analysis to
EDF scheduling. However, there is no proof to support the correctness at this moment.



\subsubsection{Improving the modeling of segmented self-suspending tasks}
\label{sec:model-interfering-improving}

In the \emph{segmented self-suspending task model}, we can simply ignore the segmentation structure of computation segments and suspension intervals 
and directly apply all the strategies for dynamic self-suspending task models. However, the analysis will become pessimistic. This is due to the fact 
that the segmented suspensions are not completely dynamic. 

%The static suspension patterns result in also certain (more predictable) suspension patterns. However, characterizing the worst-case 
%suspending patterns of the higher priority tasks to quantify the interference under the segmented self-suspending task model is not easy. 
%Similarly, one possibility is to characterize the worst-case interference in the carry-in job of a higher-priority task $\tau_i$ by analyzing 
%its self-suspending pattern, as presented in \cite{Huang:multiseg}. Another possibility is to  quantify the interference by modeling 
%it with a jitter term, as presented in \cite{RTCSA-BletsasA05}. We will explain later in Section~\ref{sec:wrong-jitter-segmented} 
%why the quantification of the interference in \cite{RTCSA-BletsasA05} is incorrect. {\bf Michael's paper in RTSS1998}. 

%The static suspension patterns result in also certain (more predictable) suspension patterns. 
Characterizing the worst-case 
suspending patterns of the higher priority tasks to quantify the interference under the segmented self-suspending task model is not easy. 
Modelling the interference by a job of a self-suspending task $\tau_i$ as multiple per-segment ``chunks", spaced apart in time by 
the respective self-suspending intervals in-between, is potentially more accurate than modelling it as a contiguous computation segment of $C_i$ units.
However, the worst-case release offset of $\tau_i$ in $hp(k)$, relative to the task $\tau_k$ under analysis, to maximize the interference needs to be identified.

To deal with this, in \cite{RTCSA-BletsasA05} the computation segments and self-suspending intervals of each interfering task are reordered, 
to create a pattern that dominates all such possible task release offsets. The computational segments of the interfering task are modelled 
as distinct tasks arriving at an offset to each other and sharing a period and arrival jitter. However, as we will explain later in 
Section~\ref{sec:wrong-jitter-segmented} the quantification of the interference in \cite{RTCSA-BletsasA05} was incorrect. 

Another possibility is to characterize the worst-case interference in the carry-in job of a higher-priority task $\tau_i$ by analyzing 
its self-suspending pattern, as presented in \cite{Huang:multiseg}. This approach does examine the different possible task release offsets
and can also be used for response time analysis compatible with Audsley's optimal priority algorithm~\cite{Audsley1991aOPA}.
Palencia and Gonz\'alez Harbour~\cite{PH:rtss98} provide another technique for modelling of interference by segmented interfering tasks, 
albeit in the context of multiprocessors. %Therefore, this work is discussed later, in Section~\ref{sec:multiprocessor-HRT}.

  
\subsection{Period Enforcement Mechanisms}   
\label{sec:release-enforce}

Self-suspension can cause substantial schedulability degradation, because the resulting non-determinism in the schedule can give rise 
to unfavourable execution patterns. To alleviate the potential impact, one possibility is to enforce periodic behaviour by 
enforcing the release time of the computation segments. There exist different categories of such enforcement mechanisms. 


\subsubsection{Dynamic online period enforcement} 
\label{sec:period-enforce}

Rajkumar \cite{Raj:suspension1991} proposed a \emph{period enforcer} 
algorithm to handle the impact of uncertain releases (such as self-suspensions). In a nutshell, the period enforcer algorithm artificially 
increases the length of certain suspensions \emph{dynamically, at run-time},
whenever a task's activation pattern carries the risk of inducing undue interference in 
lower-priority tasks. Quoting \cite{Raj:suspension1991}, the period enforcer algorithm \textit{``forces tasks to behave like ideal 
periodic tasks from the scheduling point of view with no associated scheduling penalties''}. 

The period enforcer has been revisited by Chen and Brandenburg in \cite{ChenBrandenburg}, with the following three observations:
\begin{enumerate}
	\item period enforcement can be a cause of deadline misses for self-suspending tasks sets that are otherwise schedulable;
	\item with the state-of-the-art techniques, the schedulability analysis of the period enforcer algorithm requires a task set 
	transformation which is subject to exponential time complexity; and 	
    \item the period enforcer algorithm is incompatible with all existing analyses of suspension-based locking protocols, 
	and can in fact cause ever-increasing suspension times until a deadline is missed.
\end{enumerate}

\subsubsection{Static period enforcement} 
\label{sec:static-period-enforce}

As an alternative to the online period enforcement, one may instead achieve periodicity in the activation
of computation segments and prevent the most unfavorable execution patterns from arising, by constraining
each computation segment to be released at a respective \emph{fixed offset} from its job's arrival.
These constant offsets are computed and specified \emph{offline}.
 
Suppose that the offset for the $j$-th computation  segment of task $\tau_i$ is $\phi_i^j$. This means that the 
$j$-th computation segment of task $\tau_i$ is released only at time $r_i+\phi_i^j$, where $r_i$ is the arrival time of a job 
of task $\tau_i$. That is, even if the preceding self-suspension completes before $r_i+\phi_i^j$, the computation segment in
consideration is never executed earlier. With this static enforcement, each computation segment can be represented by a sporadic 
task with a minimum inter-arrival tome $T_i$, a WCET $C_i^j$, and a relative deadline $\phi_{i,j+1}-\phi_i^j-S_i^j$ (with 
$\phi_{i,m_i+1}$ set to $D_i$.) 

Such approaches have been presented in 
\cite{RTSS-KimANR13,LR:rtas10,RTSS-ChenL14}. The method in \cite{RTSS-ChenL14} is a simple and greedy solution for 
implicit-deadline self-suspending task systems with at most one self-suspension interval per task. It assigns the 
phase $\phi_i^2$ always to $\frac{T_i+S_i^1}{2}$ and the relative deadline of the first computation segment of task $\tau_i$ to 
$\frac{T_i-S_i^1}{2}$. This is the first method in the literature with \emph{speedup factor} guarantees by using the revised relative 
deadline for earliest-deadline-first scheduling. 

The methods in \cite{RTSS-KimANR13,DBLP:journals/ieicet/DingTT09} assign each computation segment a fixed-priority level and a phase. 
Unfortunately,  in \cite{RTSS-KimANR13,DBLP:journals/ieicet/DingTT09}, the schedulability tests are not correct, and the mixed-integer 
linear programming formulation proposed in \cite{RTSS-KimANR13} is unsafe for worst-case response time guarantees. 
A detailed discussion on this matter is 
provided in Section~\ref{sec:wrong-periodic}.

\subsubsection{Slack enforcement} 
\label{sec:slack-enforce}

The slack enforcement in \cite{LR:rtas10} intends to create periodic execution enforcement 
for self-suspending tasks so that a self-suspending task behaves like an ideal periodic task.  However, as discussed in 
Section~\ref{sec:open-issues-existing}, the presented methods in \cite{LR:rtas10} require more rigorous proofs to support 
their correctness as the key lemma of the slack enforcement mechanism in \cite{LR:rtas10} is incomplete.

\subsection{Multiprocessor Scheduling for Self-Suspending Tasks}
\label{sec:multiprocessor-HRT}
  
The schedulability analysis of distributed systems is inherently similar to the schedulability analysis of multiprocessor systems following a 
\emph{partitioned} scheduling scheme. Each task is mapped on one processor and can never migrate to another processor. In~\cite{PH:rtss98}, 
Palencia and Gonz\'alez Harbour extended the worst-case response time analysis for distributed systems, and hence multiprocessor systems, to segmented self-suspending tasks. They model the effect of the self-suspending time as release jitter.
  
The first suspension-aware worst-case response time analysis for dynamic self-suspending sporadic tasks assuming a \emph{global} scheduling scheme, was presented in \cite{DBLP:conf/ecrts/LiuA13}. 
The processors are assumed to be identical and the jobs can migrate during their execution. The analysis in \cite{DBLP:conf/ecrts/LiuA13} is mainly based on the existing results in the literature for global fixed-priority and earliest deadline first scheduling for sporadic task systems without self-suspensions. Unfortunately, the schedulability test provided in \cite{DBLP:conf/ecrts/LiuA13} for global fixed-priority scheduling suffers from two errors, which were later fixed in \cite{erratu-cong-anderson}. First, the workload bound proposed in Lemma~1 (in~\cite{DBLP:conf/ecrts/LiuA13}) is unsafe. It has been acknowledged and corrected in \cite{erratu-cong-anderson} following a similar approach as in \cite{baruah2007techniques}. 
Secondly, it is optimistic to claim that there are at most $M-1$ carry-in jobs in the general case. This flaw has been inherited from an error in a 
previous work \cite{DBLP:conf/rtss/GuanSYY09}, which was pointed out and further corrected in \cite{sun2014improving,DBLP:conf/rtns/HuangC15}. 

Therefore, by adopting the analysis from \cite{DBLP:conf/rtns/HuangC15}, which is consistent with the analysis in \cite{DBLP:conf/ecrts/LiuA13}, the problem can easily be fixed. The reader is referred to \cite{erratu-cong-anderson} for further details.
  
Chen et al. \cite{ChenHLRTSS2015} studied global rate-monotonic scheduling in multiprocessor systems, including dynamic self-suspending tasks. The proposed utilization-based schedulability analysis  can easily be extended to handle constrained-deadline task systems and any given fixed-priority assignment.
  
  


%%% Local Variables:
%%% mode: latex
%%% TeX-master: "JRTS/JRTS.tex"
%%% End:


  
