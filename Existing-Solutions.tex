\section{Existing Solutions of Self-Suspending Tasks in Uniprocessor Platforms}

This section reviews the existing solutions for scheduling and analyzing the schedulability of self-suspending task models. We will first explain the commonly adopted strategies in those solutions. The general strategies are correct, but some of misconceptions were used in the literature to tackle the problem. We will provide concrete reasons and some counterexamples to explain why such misconceptions may lead to over-optimistic analysis. At the end of this section, we will provide the rule of thumb for analyzing self-suspending task systems. 

To demonstrate how the scheduling algorithms and the schedulability tests work in existing approaches, we will mainly use the following tasks in Table 1 and Table 2. For demonstrating the worst-case response time analysis, we leave some relative deadline with "?" and period $\infty$. Specifically, we will use task set ${\bf T}_1 = \setof{\tau_1, \tau_2, \tau_3}$, ${\bf T}_2 = \setof{\tau_1, \tau_2, \tau_3, \tau_4}$, ${\bf T}_3 = \setof{\tau_\alpha, \tau_\beta, \tau_\gamma}$ in our examples. Unless specified, we will implicitly assume that these three example task sets are scheduled under Rate-Monotonic scheduling. 

\begin{table} 
    \begin{tabular}{|c|c|c|c|c|}
        & $C_i$ &  $S_i$&  $D_i$ & $T_i$\\ 
        \hline
        $\tau_\alpha$ & 1 & 0 &  2 & 2\\ 
        $\tau_\beta$ &  5&  5& 20 & 20 \\ 
        $\tau_\gamma$ & 1 & 0  & ? & $\infty$ \\ 
        \hline
    \end{tabular} 
    \caption{Examples for dynamic self-suspending tasks}
\end{table}

\begin{table} 
    \begin{tabular}{|c|c|c|c|}
        & $(C_i^1, S_i^2, C_i^2)$ &  $D_i$ & $T_i$\\ 
        \hline
        $\tau_1$ & (2, 0, 0) &  5 & 5\\ 
        $\tau_2$ &  (2, 0, 0) & 10 & 10 \\ 
        $\tau_3$ & (1, 5, 1) & 15  & 15\\
        $\tau_4$ & (3, 0, 0) & ? & $\infty$\\
        \hline
    \end{tabular} 
    \caption{Examples for dynamic segmented-suspending tasks}
\end{table}


\subsection{Common Strategies}

For self-suspending sporadic task systems, while executing, a job may suspend itself or even must suspend itself in the segmented self-suspension model. While a job suspends, the scheduler removes the job from the ready queue. Such suspensions should be well characterized and the resulting workload interference should be well quantified to analyze the schedulability of the task systems. 

{\bf an example}

There are some common strategies to characterize and quantify the impact due to self-suspensions:
\begin{itemize}
\item {\bf Convert All Self-Suspension into Computation:} This is the simplest and the most pessimistic strategy. It basically converts all self-suspending time into computation time. That is, we can consider that the execution time of task $\tau_i$ is always $C_i+S_i$. After the conversion, we only have sporadic real-time tasks. Therefore, all the existing results for sporadic task systems can be adopted. The proof can be done with the following simple interpretation: The suspension of a job may make the processor idle. If two jobs suspend at the same time and the processor idles in a certain time interval in the actual schedule, it can be imagined that one of these two jobs have shorter execution time (than its worst-case execution time $C_i+S_i$). Such earlier completion does not affect the schedulability analysis. Therefore, putting $C_i+S_i$ as the worst-case execution time for every task $\tau_i$ is a very safe analysis for both dynamic- and static-scheduling policies.  Such an approach has been widely used as the baseline of more accurate analyses in the literature.

With this schedulability test, it is easy to see that none of the three example task sets ${\bf T}_1$, ${\bf T}_2$, ${\bf T}_2$ cannot be classified as feasible since $\frac{1}{2} + \frac{5+5}{20} + \frac{1}{D_\gamma} > 1$ and $\frac{2}{5}+\frac{2}{10}+\frac{1+5+1}{15} > 1$.

\item {\bf Convert Higher-Priority Tasks into Sporadic Tasks:} In static-priority scheduling, we can convert the higher-priority self-suspending tasks into equivalent sporadic real-time tasks: When we analyze the schedulability of a task $\tau_k$, we can convert the higher-priority self-suspending tasks into sporadic tasks by treating the suspension as computation. That is, a higher-priority task $\tau_i$ (higher than task $\tau_k$) has now worst-case execution time $C_i+S_i$. This simplifies the analysis. After converting, we only have one self-suspending task left as the lowest-priority task in the system.  Such a conversion is useful for analyzing segmented self-suspending task model. However, such a conversion is not very useful for analyzing dynamic self-suspending task models, since we have to consider the worst case that the self-suspension of task $\tau_k$ makes the processor idle. Therefore, we also have to convert $\tau_k$'s self-suspension into computation. This results in identical analysis by converting self-suspension into computation for all the tasks. 

With the conversion, the fundamental problem is to analyze the worst-case response time of a self-suspending task $\tau_k$ as the lowest-priority task in the task system, when all the other higher priority tasks are ordinary sporadic real-time tasks. One simple strategy is to analyze the worst-case response time $R_k^j$ for each computation segment $C_k^j$. The schedulability test of task $\tau_k$ then is to simply verify whether $R_k^{m_k} + \sum_{j=1}^{m_k-1} R_k^j + S_k^j \leq D_k$.


The problem with only one self-suspending task as the lowest-priority task has been specifically studied in \cite{LR:rtas10,ecrts15nelissen}. Unfortunately, the analysis in \cite{LR:rtas10} is flawed. We will explain the reasons in Section~\ref{sec:wrong-critical}.




\item {\bf Quantify Additional Interference due to Self-Suspensions:} Suspension may result in more workload from higher-priority jobs to interfere with a lower-priority job. This strategy is to convert the suspension time of a job of task $\tau_k$ under analysis into computation. Suppose that this job under analysis arrives at time $t_k$. The other higher-priority jobs except the job under analysis are considered to (possibly) have self-suspensions. This is the completely opposite strategy to the previous strategy. Since a higher-priority self-suspending job may suspend itself before $t_k$ and resume after $t_k$, the self-suspending behaviour of a task $\tau_i$ can be considered to bring \emph{at most} one \emph{carry-in} job to be \emph{partially} executed after $t_k$ if $D_i \leq T_i$. As we have converted task $\tau_k$'s self-suspension time into computation, the finishing time of the job of task $\tau_k$ is the earliest moment after $t_k$ such that the processor idles. 
\begin{itemize}
\item In the \emph{dynamic self-suspending task model}, the above analysis implies that the higher-priority jobs arrived after time $t_k$ \emph{should not} suspend themselves to create the maximum interference. Therefore, suppose that the first arrival time of task $\tau_i$ after $t_k$ is $t_i$, i.e., $t_i \geq t_k$. Then, the demand of task $\tau_i$ released at time $t \geq t_i$ is $\ceiling{\frac{t-t_i}{T_i} } C_i$. So, we just have to account the demand of the carry-in job of task $\tau_i$ executed between $t_k$ and $t_i$. The workload of the carry-in job can be up to $C_i$, but can also be characterized in a more precise manner. The approaches in this category are presented in \cite{huangpass:dac2015,LiuChen:rtss2014} by greedily counting $C_i$ in the carry-in job. Jane W.S. Liu in her textbook \cite[Page 164-165]{Liu:2000:RS:518501} presents an approach to quantify the higher-priority tasks by setting up the \emph{blocking time} induced by self-suspensions. In her analysis, a job of task $\tau_k$ can suffer from the \emph{extra delay} due to self-suspending behavior as a factor of blocking time, denoted as $B_k$, as follows: (1) The blocking time contributed from task $\tau_k$ is $S_k$. (2) A higher-priority task $\tau_i$ can only block the execution of task $\tau_k$ by at most $b_i=min(C_i, S_i)$ time units. In the textbook \cite{Liu:2000:RS:518501}, the blocking time $B_k=S_k+\sum_{i=1}^{k-1} b_i$ is then used to perform utilization-based analysis for rate-monotonic scheduling. However, there was no proof in the textbook. Fortunately, the recent report from Chen [XXX] has provided a proof to support the correctness of the above method in \cite{Liu:2000:RS:518501}.

\vspace{0.1in}


Another way to quantify the impact is to model the impact of the carry-in job by using the concept of \emph{jitter}. If the jitter of task $\tau_i$ to model self-suspension is $J_i$, then, the demand of task $\tau_i$ released from $t_i-T_i$ up to time $t+t_k$  (i.e., the demand that can be executed from $t_k$ to $t_k+t$) is $\ceiling{\frac{t+J_i}{T_i} } C_i$. A safe way it to set $J_i$ to $T_i$, which can be imagined as a pessimistic analysis by assuming that the carry-in job of task $\tau_i$ has execution time $C_i$ and the release time $t_i$ is $t_k$. A more precise way to quantify the jitter is to use the worst-case response time of a higher-priority task $\tau_i$. Therefore, we can set the jitter $J_i$ of task $\tau_i$ to $T_i-C_i$ \cite{huangpass:dac2015,Raj:suspension1991} or  $R_i-C_i$ \cite{huangpass:dac2015}, where $R_i$ is the worst-case response time of a higher-priority task $\tau_i$. There have been some flawed analyses in the literature \cite{ECRTS-AudsleyB04,RTAS-AudsleyB04,RTCSA-KimCPKH95} which quantify the jitter of task $\tau_i$ by setting $J_i$ to $S_i$. We will explain later in Section \ref{sec:wrong-jitter} why setting $J_i$ to $S_i$ is in general too optimistic. 

\item In the \emph{segmented self-suspending task model}, we can simply ignore the segmentation structure of computation segments and suspension intervals and directly apply all the strategies for dynamic self-suspending task models. However, the analysis will become pessimistic. This is due to the fact that the segmented-suspensions are not completely dynamic. The static suspension patterns result in also certain (more predictable) suspension patterns. However, characterizing the worst-case suspending patterns of the higher priority tasks to quantify the additional interference under segmented self-suspending task model is not easy. Similarly, one possibility is to characterize the worst-case interference in the carry-in job of a higher-priority task $\tau_i$ by analyzing its self-suspending pattern, as presented in \cite{Huang:multiseg}. Another possibility is to  quantify the interference by modeling it with a jitter term, as presented in \cite{RTCSA-BletsasA05}. We will explain later in Section~\ref{sec:wrong-carryin} why the quantification of the interference in \cite{RTCSA-BletsasA05} is incorrect. {\bf Michael's paper in RTSS1998}.
\end{itemize}

\item {\bf Handle Self-Suspension Segments of the Task under Analysis:} Greedily converting the suspension time of a job of task $\tau_k$ under analysis into computation can also become very pessimistic if $S_k$ is much larger than $C_k$. However, the decision to convert a task $\tau_k$ has to be done carefully. Now, we can consider a simple example to analyze the worst-case response time of task $\tau_k = ((C_k^1, S_k^1, C_k^2), T_k, D_k)$. We can have two options:
\begin{itemize}
\item Convert $S_k^2$ into computation, and then apply the above analysis by considering that task $\tau_k$ has execution time $C_k^1+S_k^1+C_k^2$. We simply have to verify whether the worst-case response time is no more than $D_k$.
\item Treat each of the computation segments of task $\tau_k$ individually by applying the worst-case higher-priority interference, regardless of its previous computation segments. We need to verify if the suspension time $S_k$ plus sum of the worst-case response time of all the computation segments of task $\tau_k$ is no more than $D_k$. 
\end{itemize}
The benefit of the former approach is due to that it only pessimistically counts the additional higher-priority interference once. However, it also suffers from the pessimism by converting $S_k^1$ into computation. The benefit of the latter approach is due to the fact that the suspension time is not over-counted as computation. However, it also over-counts the carry-in workload since every computation segment may have to pessimistically count the worst-case workload of the carry-in jobs. Both of these two approaches are adopted in the literature \cite{ecrts15nelissen,Huang:multiseg,RTCSA-BletsasA05}. They can be both applied and the better result is returned.

\item {\bf Enforce Periodic Behaviour by Release Time Enforcement:}   Self-suspension can cause substantial schedulability degradation. To leviate the impact on additional interference due to self-suspension, one possibility is to enforce the periodic behaviour by enforcing the release time of the computation segments. There are two categories of such enforcement. 
  \begin{itemize}
  \item {\it Use resource reservation servers}: Rajkumar \cite{Raj:suspension1991} proposes a \emph{period enforcer} algorithm to handle the impact of uncertain releases (like self-suspensions). (One can imagine that a sporadic server \cite{RTSS-SpruntLS88} with capacity $C_k$ and replenishment period $T_k$ is the reservation server to run task $\tau_k$, by handling the self-suspension.) The period enforcer algorithm was shown the have a good property: ``A deferrable task that is schedulable under its worst-case conditions is also schedulable under the period enforcer algorithm.'' in Theorem 3.5 in \cite{Raj:suspension1991}. If the above statement is correct, then handling the execution of self-suspending sporadic tasks can be very easy and effective.
 However, we will provide a counter example in Section~\ref{sec:wrong-periodic} for the above statement in Section \cite{Raj:suspension1991}.

%Therefore, we can use the period enforcer algorithm to verify the schedulability of task $\tau_k$ under static-priority scheduling. However, since there is no schedulability test in \cite{Raj:suspension1991}, how to achieve this remains an open problem.
  \item {\it Set a constant offset to constrain the release time of a computation segment}: Suppose that the offset for the $j$-th computation segment of task $\tau_i$ is $\phi_i^j$. This means that the $j$-th computation segment of task $\tau_i$ is released only at time $r_i+\phi_i^j$, in which $r_i$ is the arrival time of a job of task $\tau_i$. With the enforcement, each computation segment can be represented by a sporadic task with a period $T_i$, a WCET $C_i^j$, and a relative deadline $\phi_{i,j+1}-\phi_i^j-S_i^j$. (Here, $\phi_{i,m_i+1}$ is set to $D_i$.) Such approaches have been presented in \cite{RTSS-KimANR13,LR:rtas10,RTSS-ChenL14}. The method in \cite{RTSS-ChenL14} is a simple greedy solution for implicit-deadline self-suspending task systems with at most one self-suspension interval per task. It assigns the phase $\phi_i^2$ always to $\frac{T_i+S_i^1}{2}$ and the relative deadline of the first computation segment of task $\tau_i$ to $\frac{T_i-S_i^1}{2}$. This is the first method in the literature with \emph{speedup factor} guarantees by using the revised relative deadline for earliest-deadline-first scheduling. The method in \cite{RTSS-KimANR13} assigns each computation segment a static-priority level and a phase. Unfortunately,  in \cite{RTSS-KimANR13}. the schedulability test is not correct, and the proposed mixed-integer linear programming is unsafe for worst-case response time guarantees. The method in \cite{LR:rtas10} is XXX (left to Geoffrey...)
  \end{itemize}
\item {\bf Special Cases with Good Observations:}
\end{itemize}


\subsection{Misconceptions in Some Existing Results}
\subsubsection{Incorrect Assumptions in Critical Instant Theorem with Synchronous Releases}
\label{sec:wrong-critical}

\subsubsection{Incorrect Quantifications of Additional Interferences due to Carry-In Jobs}
\label{sec:wrong-carryin}

\subsubsection{Incorrect Quantifications of Jitter}
\label{sec:wrong-jitter}


\subsubsection{Incorrect Periodic Execution Enforcement}
\label{sec:wrong-periodic}

\subsection{Rule of Thumb When Considering Self-Suspending Systems}
  
  
  
  
  
  
  
  
  
  
  
  
  
  
  
  
  




%%% Local Variables:
%%% mode: latex
%%% TeX-master: "authorea_build/authorea_paper.tex"
%%% End:

  