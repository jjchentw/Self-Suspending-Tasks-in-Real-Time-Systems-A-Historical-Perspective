\section{Open Issues in the Existing Results}

We have carefully re-examined the results related to self-suspending
real-time tasks in the literature in the past 25 years. However, there
are also some results in the literature that may require further
elaborations. These include the following results in the literature
\begin{itemize}
\item Devi (in Theorem 8 in \cite[Section
  4.5]{DBLP:conf/ecrts/Devi03}) extends the analysis proposed by Jane
  W.S. Liu in her textbook \cite[Page 164-165]{Liu:2000:RS:518501} to
  EDF scheduling. This method quantifies the additional interference
  due to self-suspensions from the higher-priority jobs by setting up
  the \emph{blocking time} induced by self-suspensions. However, there
  is no formal proof in \cite{DBLP:conf/ecrts/Devi03}. The proof made
  by Chen et al. in \cite{ChenHuangNelissen} for fixed-priority
  scheduling cannot be directly extended to EDF scheduling. The
  correctness of Theorem 8 in \cite[Section
  4.5]{DBLP:conf/ecrts/Devi03} should be supported with a rigorous
  proof, since self-suspension behaviour has induced several
  non-trivial phenomena.

\item For segmented self-suspending task systems with at most one
  self-suspending interval, Lakshmanan and Rajkumar propose two slack
  enforcement mechanisms in \cite{LR:rtas10} to shape the demand of a
  self-suspending task so that the task behaves like an ideal periodic
  task.  From the scheduling point of view, this means that there is
  no \emph{potentially} scheduling penalty when analyzing the interferences of the
  higher-priority tasks. (But, the suspension time of the task under
  analysis has to be converted into computation.) The correctness of the dynamic slack
  enforcement in \cite{LR:rtas10} is heavily based on the statement in Lemma
  4 in \cite{LR:rtas10}. However, the proof is not rigorous with the
  following reasons:
  \begin{itemize}
  \item Firstly, the proof argues: ``\emph{Let the duration $R$ under
    consideration start from time $s$ and finish at time $s +
    R$. Observe that if $s$ does not coincide with the start of the
    Level-$i$ busy period at $s$, then $s$ can be shifted to the left
    to coincide with the start of the Level-$i$ busy period. Doing so
    will not decrease the Level-$i$ interference over $R$.}'' This
    argument has to be proved by also handling cases in which a task
    may suspend before the $Level-i$ busy period. This results in the
    possibility that a higher-priority task $\tau_j$ starts with the
    second computation segment in the Level-$i$ busy
    period. Therefore, the first and the third paragraphs in the proof
    of Lemma 4 \cite{LR:rtas10} require a more rigorous flow.

  \item Secondly, the proof argues: ``\emph{The only property
      introduced by dynamic slack enforcement is that under worst-case
      interference from higher-priority tasks there is no slack
      available to $J_j^p$ between $f_j^p$ and $\rho_j^p + R_j$.
      $\ldots$ The second segment of $\tau_j$ is never delayed under
      this transformation, and is released sporadically.} '' In fact,
    the slack enforcement may make the second computation segment
    arrive earlier than its worst-case. For example, we can greedily
    start with the worst-case interference of task $\tau_j$ in the
    first iteration, and do not release the higher-priority tasks
    (higher than $\tau_j$) after the arrival of the second job of task
    $\tau_j$. This can immediately create some release jitter of the
    second computation segment $C_j^2$.
  \end{itemize}
  With the same reasons, the static slack enforcement algorithm in
  \cite{LR:rtas10} also requires a more rigorous proof.
\end{itemize}




\section{Summary and Conclusions}
  
 Self-suspending behaviour is becoming an increasingly prominent
characteristic in real-time systems such as: (i) I/O-intensive systems
(ii) multi-processor synchronization and scheduling, and (iii)
computation offloading with coprocessors, like Graphics Processing
Units (GPUs).  This paper reviews the literature in the light of
recent developments in the analysis of self-suspending tasks,
explains the general methodologies, summarizes the hardness and
the complexity classes, and points out several 
misconceptions in the literature for this topic. We
explain concrete examples to demonstrate the effect of these
misconceptions, list the flawed statements in the literatures, and
present the potential solutions. These misconceptions include:
\begin{itemize}
\item Incorrect quantifications of jitter for dynamic self-suspending
  task systems, which was used in
  \cite{ECRTS-AudsleyB04,RTAS-AudsleyB04,RTCSA-KimCPKH95}.  This
  misconception was unfortunately adopted in
  \cite{zeng-2011,bbb-2013,yang-2013,kim-2014,han-2014,carminati-2014,yang-2014,lakshmanan-2009} to analyze the worst-case response time for
  partitioned multiprocessor real-time locking protocols.
\item Incorrect quantifications of jitter for dynamic self-suspending
  task systems, which was used in  \cite{RTCSA-BletsasA05}.
\item Incorrect assumptions in the critical instant with
  synchronous releases, which was used in \cite{LR:rtas10}.
\item Counting highest-priority self-suspending time to reduce the
  interference, which was used in  \cite{RTSS-KimANR13}. 
\item Incorrect segmented fixed-priority scheduling with periodic
  enforcement, which was used in \cite{RTSS-KimANR13,DBLP:journals/ieicet/DingTT09}.
\end{itemize}
For completeness, the above misconceptions are listed in Table \jj{XXX}.

In order to make all the statements in this review rigorous, several
individual reports are filed by the subset of the authors. These
include the proof \cite{ChenHuangNelissen} of the correctness of the
analysis from Jane W.S. Liu in her textbook \cite[Page
164-165]{Liu:2000:RS:518501}, the re-examination and the limitations
\cite{ChenBrandenburg} of the period enforcer algorithm proposed in
\cite{Raj:suspension1991}, the erratum report \cite{BletsasReport2015}
of the misconceptions in
\cite{ECRTS-AudsleyB04,RTAS-AudsleyB04,RTCSA-BletsasA05}, the updated
erratum \cite{erratu-cong-anderson} of the unsafe analysis in
\cite{erratu-cong-anderson}, and the erratum \jj{XXXX} of the
misconceptions in \cite{RTSS-KimANR13}.  In order not to make this
review too lengthy, these reports and errata are shortly summarized in
this review. We encourage the readers to refer to these reports and
errata for more detailed explanations.


%%% Local Variables:
%%% mode: latex
%%% TeX-master: "JRTS/JRTS.tex"
%%% End:
