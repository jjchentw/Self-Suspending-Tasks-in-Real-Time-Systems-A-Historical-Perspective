\section{Open Issues and Summary}
\subsection{Summary}  
 Self-suspending behaviour is becoming an increasingly prominent
characteristic in real-time systems such as: (i) I/O-intensive systems
(ii) multi-processor synchronization and scheduling, and (iii)
computation offloading with coprocessors, like graphics processing
units (GPUs).  This paper has reviewed the literature in the light of
recent developments in the analysis of self-suspending tasks,
explained the general methodologies, summarized the computational complexity classes, 
and detailed a number of 
misconceptions in the literature concerning this topic. We
have given concrete examples to demonstrate the effect of these
misconceptions, listed some flawed statements in the literature, and
presented potential solutions. These misconceptions include:
\begin{itemize}
\item Incorrect quantification of jitter for dynamic self-suspending
  task systems, which was used in
  \cite{ECRTS-AudsleyB04,RTAS-AudsleyB04,RTCSA-KimCPKH95,MingLiRTCSA1994}.  This
  misconception was unfortunately adopted in
  \cite{zeng-2011,bbb-2013,yang-2013,kim-2014,han-2014,carminati-2014,yang-2014,lakshmanan-2009} to analyze the worst-case response time for
  partitioned multiprocessor real-time locking protocols.
\item Incorrect quantification of jitter for segmented self-suspending
  task systems, which was used in  \cite{RTCSA-BletsasA05}.
\item Incorrect assumptions in the critical instant with
  synchronous releases, which was used in \cite{LR:rtas10}.
\item Incorrectly counting highest-priority self-suspending time to reduce the
  interference, which was used in  \cite{RTSS-KimANR13}. 
\item Incorrect segmented fixed-priority scheduling with periodic
  enforcement, which was used in \cite{RTSS-KimANR13,DBLP:journals/ieicet/DingTT09}.
\end{itemize}
For completeness, all the misconceptions, open issues, closed issues,
and inherited flaws discussed in this paper are listed in Table~\ref{tab:summary}.

This review extensively references errata and reports as follows:
the proof \cite{ChenHuangNelissen} of the correctness of the
analysis by Jane W.S. Liu in her book \cite[Page
164-165]{Liu:2000:RS:518501}; the re-examination and the limitations
\cite{ChenBrandenburg} of the period enforcer algorithm proposed in
\cite{Raj:suspension1991}; the erratum report \cite{BletsasReport2015}
of the misconceptions in
\cite{ECRTS-AudsleyB04,RTAS-AudsleyB04,RTCSA-BletsasA05}; and the erratum \cite{Kim2016} of the
misconceptions in \cite{RTSS-KimANR13}. 
For brevity, these errata and reports are only summarized in
this review. We encourage the readers to refer to these reports and
errata for more detailed explanations.



\begin{table}[t]
  \centering
\scalebox{0.8}{
\begin{tabular}{||C{3cm}|L{7cm}||C{3cm}|C{2cm}|}  
  \hline
  \hline
  Type of Arguments& Affected papers and statements &
  Potential Solutions & (flaw/issue) status\\
  \hline
  \multirow{6}{*}{Conceptual Flaws} & \cite{ECRTS-AudsleyB04,RTAS-AudsleyB04}: Wrong
  quantification of jitter& See Section~\ref{sec:wrong-jitter-dynamic} or the
  erratum filed by the authors \cite{BletsasReport2015}& solved\\
  \cline{2-4}
& \cite{MingLiRTCSA1994}:   Wrong quantification of jitter & See Section
\ref{sec:wrong-jitter-dynamic} & solved\\
  \cline{2-4}
& \cite{RTCSA-BletsasA05}:   Wrong quantification of jitter & See Section
\ref{sec:wrong-jitter-segmented} or \cite{BletsasReport2015}& solved\\
  \cline{2-4}
& \cite{LR:rtas10}:  Critical instant theorem in Section III and the
response time analysis are incorrect & See Section
\ref{sec:wrong-critical} or \cite{ecrts15nelissen} & solved\\
\cline{2-4}
  & \cite{RTSS-KimANR13}: Theorems 2 and 3 are incorrect & See
  Section~\ref{sec:wrong-highest-priority} & solved\\
  \cline{2-4}
  & \cite{RTSS-KimANR13}: Section IV is
  incorrect & See Section~\ref{sec:wrong-periodic} &
  not solved\\
  \cline{2-4}
  & \cite{DBLP:journals/ieicet/DingTT09}: Section 3 is
  incorrect & See Section~\ref{sec:wrong-periodic} &
  not solved\\
  \cline{2-4}
\hline
\multirow{2}{*}{Inherited Flaws} &
\cite{RTCSA-KimCPKH95,zeng-2011,bbb-2013,yang-2013,kim-2014,han-2014,carminati-2014,yang-2014,lakshmanan-2009}:
 Adopting wrong quantifications of jitters (refer to
 Section~\ref{sec:syn} in this paper)& See
Section~\ref{sec:safe_bound} & solved\\
\cline{2-4}
& \cite{DBLP:conf/ecrts/LiuA13}: Inherited flaw from
\cite{DBLP:conf/rtss/GuanSYY09} and unsafe Lemma 3 to quantify the workload& See the erratum
\cite{erratu-cong-anderson} filed by the authors & solved\\
\hline
\multirow{2}{*}{Closed Issues} & \cite[Page
164-165]{Liu:2000:RS:518501}: schedulability test without any proof & See
\cite{ChenHuangNelissen} for the proof& solved\\
\cline{2-4}
 &\cite{Raj:suspension1991}: period enforcer can be used for
 deferrable task systems. It may result in deadline
 misses for self-suspending tasks and is not compatible with multiprocessor synchronization & See
 \cite{ChenBrandenburg} for the explanations.& solved\\
\hline
\multirow{2}{*}{Open Issues} &  \cite{DBLP:conf/ecrts/Devi03}: Proof
of Theorem
8  is incomplete& ?& ?\\
\cline{2-4}
 & \cite{LR:rtas10}: Proofs for slack enforcement in Sections IV and V
 are incomplete & ?& ?\\
  \hline
  \hline
\end{tabular}}
\vspace{0.1in}
  \caption{List of flaws/incompleteness and their solutions in the
    literature. All the references to Section X in the column
    ``Potential Solutions'' are listed for this paper.}
  \label{tab:summary}
\end{table}


\subsection{Open Issues in the Existing Results}
\label{sec:open-issues-existing}  
We have carefully re-examined the results related to self-suspending
real-time tasks in the literature in the past 25 years. However, there
are also some results in the literature that may require further
elaborations. These include the following results in the literature:
\begin{itemize}
\item Devi (in Theorem 8 in \cite[Section
  4.5]{DBLP:conf/ecrts/Devi03}) extended the analysis proposed by Jane
  W.S. Liu in her book \cite[Page 164-165]{Liu:2000:RS:518501} to
  EDF scheduling. This method quantifies the additional interference
  due to self-suspensions from the higher-priority jobs by setting up
  the \emph{blocking time} induced by self-suspensions. However, there
  is no formal proof in \cite{DBLP:conf/ecrts/Devi03}. The proof made
  by Chen et al. in \cite{ChenHuangNelissen} for fixed-priority
  scheduling cannot be directly extended to EDF scheduling. The
  correctness of Theorem 8 in \cite[Section
  4.5]{DBLP:conf/ecrts/Devi03} should be supported with a rigorous
  proof, since self-suspension behaviour has induced several
  non-trivial phenomena.

\item For segmented self-suspending task systems with at most one
  self-suspending interval, Lakshmanan and Rajkumar proposed two slack
  enforcement mechanisms in \cite{LR:rtas10} to shape the demand of a
  self-suspending task so that the task behaves like an ideal ordinary
  periodic
  task.  From the scheduling point of view, this means that there is
  no \emph{potential} scheduling penalty when analyzing the interferences of the
  higher-priority tasks. (But, the suspension time of the task under
  analysis has to be converted into computation.) The correctness of the dynamic slack
  enforcement in \cite{LR:rtas10} is heavily based on the statement of Lemma
  4 in \cite{LR:rtas10}. However, the proof is not rigorous for the
  following reasons:
  \begin{itemize}
  \item Firstly, the proof argues: ``\emph{Let the duration $R$ under
    consideration start from time $s$ and finish at time $s +
    R$. Observe that if $s$ does not coincide with the start of the
    Level-$i$ busy period at $s$, then $s$ can be shifted to the left
    to coincide with the start of the Level-$i$ busy period. Doing so
    will not decrease the Level-$i$ interference over $R$.}'' This
    argument has to be proved by also handling cases in which a task
    may suspend before the $Level-i$ busy period. This results in the
    possibility that a higher-priority task $\tau_j$ starts with the
    second computation segment in the Level-$i$ busy
    period. Therefore, the first and the third paragraphs in the proof
    of Lemma 4 \cite{LR:rtas10} require more rigorous reasoning.

  \item Secondly, the proof argues: ``\emph{The only property
      introduced by dynamic slack enforcement is that under worst-case
      interference from higher-priority tasks there is no slack
      available to $J_j^p$ between $f_j^p$ and $\rho_j^p + R_j$.
      $\ldots$ The second segment of $\tau_j$ is never delayed under
      this transformation, and is released sporadically.} '' In fact,
    the slack enforcement may make the second computation segment
    arrive earlier than its worst-case. For example, we can greedily
    start with the worst-case interference of task $\tau_j$ in the
    first iteration, and do not release the higher-priority tasks
    (higher than $\tau_j$) after the arrival of the second job of task
    $\tau_j$. This can immediately create some release jitter of the
    second computation segment $C_j^2$.
  \end{itemize}
  With the same reasons, the static slack enforcement algorithm in
  \cite{LR:rtas10} also requires a more rigorous proof.
\end{itemize}

\subsection{Potentially Correct Approaches}
\label{sec:correct-solutions} 

We would like to conclude this review by providing some positive notes on the available results on the design and analyses of hard real-time systems
involving self-suspending tasks.  At the date of the writing of this
document, no concern has been raised regarding the correctness of the following results. % yet; therefore, they are potentially correct.
\begin{itemize}
\item For segmented self-suspending task systems: 
  \begin{enumerate}
  \item Rajkumar's period enforcer \cite{Raj:suspension1991}
    if a self-suspending task
    can only suspend at most once before any computation starts;
  \item the result by Palencia and Gonz\'alez
    Harbour~\cite{PH:rtss98} using the arrival jitter of a
    higher-priority task properly with an offset (also for
    multiprocessor partitioned scheduling);
  \item the proof of ${\cal NP}$-hardness in the strong sense to find a feasible
    schedule and negative results with respect to the speedup factors,
    provided by Ridouard, Richard, and Cottet \cite{Ridouard_2004};
  \item the result by Nelissen et al. \cite{ecrts15nelissen} by
    enumerating the worst-case interference from higher-priority
    sporadic tasks with an exhaustive search;
  \item the result by Chen and Liu~\cite{RTSS-ChenL14} and Huang and
    Chen \cite{WC16-suspend-DATE} by using the release-time
    enforcement as described in
    Section~\ref{sec:static-period-enforce};\footnote{Chen and Liu
      found a typo in Theorem 3 in \cite{RTSS-ChenL14}
      and filed a corresponding erratum in their websites.}
  \item the result by Huang and Chen \cite{Huang:multiseg} by
    exploring the priority assignment problem and analyzing the
    carry-in computation segments together.
  \end{enumerate}
\item For dynamic self-suspending task systems on uniprocessor
  platforms:
  \begin{enumerate}
  \item the analysis provided in~\cite[Pages
    164-165]{Liu:2000:RS:518501} by Liu based on the proof in \cite{ChenHuangNelissen}; 
  \item the utilization-based analysis by Liu and Chen
    \cite{LiuChen:rtss2014} under rate-monotonic scheduling;
  \item the priority assignment and the schedulability analysis with a
    speedup factor $2$, with respect to the optimal fixed-priority
    scheduling, by Huang et
    al. \cite{huangpass:dac2015};
  \end{enumerate}
\item For dynamic self-suspending task systems on homogeneous multiprocessor
  platforms:
  \begin{enumerate}
  \item the schedulability test for global EDF scheduling by Liu and
    Anderson \cite{DBLP:conf/ecrts/LiuA13};
  \item the schedulability test by Liu et
    al. \cite{DBLP:conf/ecrts/LiuCH014} for harmonic task
    systems with strictly periodic job arrivals;
  \item the utilization-based schedulability analysis by Chen, Huang,
    and Liu \cite{ChenHLRTSS2015} by considering carry-in jobs as
    bursty behaviour.
  \end{enumerate}
\end{itemize}
The solutions and fixes listed in Table~\ref{tab:summary} for the
affected papers and statements seem to be correct.


% \subsection{Open Issues for Future Researches}
% \label{sec:open-issues-future}

% We would also like to point out some possible future research directions for
% scheduling self-suspending task systems. 
% \begin{itemize}

% \item For segmented self-suspending tasks, the negative results presented in \cite{Ridouard_2004} hold for arbitrary suspension delays. Precisely, there is no relationship between the task execution time and its suspension delay. An interesting open issue is to study if positive results can be obtained if constrained-suspension delays are considered (\ie, the suspension delay of a task does not exceed its execution time: $S_i \leq C_i$, for every task $\tau_i$).
% \end{itemize}


%%% Local Variables:
%%% mode: latex
%%% TeX-master: "JRTS/JRTS.tex"
%%% End:
